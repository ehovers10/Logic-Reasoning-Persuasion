\documentclass[10pt]{article}

\usepackage[letterpaper]{geometry}
\usepackage{enumerate,verbatim,mdwlist}
\usepackage{fancyhdr}
\usepackage{linguex}
\usepackage[colorlinks=true,linkcolor=blue]{hyperref}

%Packages needed for trees
\usepackage{amsfonts,amsmath,amssymb}
\usepackage[varg]{txfonts}
\usepackage{qtree}

%% Margin Setting
\geometry{hmargin={.5in,.5in},vmargin={1in,1in}}
\setlength{\parindent}{0.0in}
\setlength{\parskip}{3mm}
\setlength{\tabcolsep}{10pt}
\setlength{\arraycolsep}{10pt}

%% Header
\setlength{\headheight}{23pt}
\pagestyle{fancy}
\fancyhead{}
\fancyhead[L]{Phil 101, f14}
\fancyhead[C]{Homework \# 2}
\fancyhead[R]{Due: Wednesday October 1, 2014 \\ Point total: 20 points}

\begin{document}

\textbf{Name:}\underline{  Answer key  }

\paragraph{Directions:} For each of the passages below, determine what, if any, informal fallacy the argument commits.  It can be any of the fallacies that we discussed in class, but some passages may not commit a fallacy at all.  You only need to write down one of the names for the appropriate fallacy, but if you provide a short explanation of your answer, I will take it into account in assigning partial credit.  You may find it helpful to reconstruct the argument in premise/conclusion form to better bring out the fallacy involved. (\textbf{2 points each})

\begin{enumerate}
  \setlength\itemsep{25pt}

  \item ``There's no reason for you to help out at the soup kitchen.  After all, the hungry people there are a part of the total problem of world hunger.  And world hunger is a huge problem that no person can solve on their own. So, your efforts at the soup kitchen won't do any good.''
  
  \textit{This commits the fallacy of division, which is a fallacy of illicit transference.  The whole of world hunger is said to have the property of being insoluble, and this property is illicity transfered to the part of hunger at a single soup kitchen.}
  \\
  
  \underline{Additionally receiving credit:} 
 \begin{description}
  \item [Full credit] Red herring, weak analogy
  \item [Half credit] Straw man, false cause (if explanation given)
 \end{description}


  \item ``The teacher asked me to pass the exam out to all the students, which I did.  Since I passed the exam, I think I deserve an A in the course.''
  
  \textit{This commits the fallacy of equivocation, which is a fallacy of presumption.  The word ``passed'' is used in two different senses. The first sense has nothing to do with one's grade in the course.}
  \\
  
  \underline{Additionally receiving credit:}
 \begin{description}
  \item [Full credit] False cause
  \item [Half credit] No fallacy, red herring, appeal to pity (if explanation given)
 \end{description}
  
  \item ``The governor says we need to alleviate overcrowding in prisons by releasing some of the prisoners.  But prisons are there for a reason.  There are serious criminals out there that make all of us afraid to be out on the streets.  Basically, the governor wants to turn us into a crime state. His plan is crazy!''

 \textit{This commits the fallacy of straw man, which is a fallacy of relevance.  The governor's plan is reduced to one of creating chaos, and that is easily dismissed with.  But the governor may have a stronger case that is not even addressed.}
 
 This is \textbf{not} an \textit{ad hominem} argument.  The governor's \textit{plan} is addressed on its merits, even if its better merits are ignored.  And it is \textbf{not} a \textit{red herring}.  Issues of public safety are not off topic for this argument.
 \\
  
  \underline{Additionally receiving credit:} 
 \begin{description}
  \item [Full credit] Hasty generalization, appeal to the people
 \end{description}
  
  \item ``Same-sex marriage should never be allowed in this state. If we allow gays to marry each other, then in no time uncles will marry their nephews and nieces. Then fathers will marry their daughters, mothers will marry their sons, and brothers will marry their sisters. Before long, pet owners will marry their dogs and cats, and this will lead to the complete destruction of civilized life.''
  
\textit{This commits the fallacy of slippery slope, which is a fallacy of weak induction.  The arguer posits a chain of events that will result from allowing same-sex marriage, with the final event leaving us in disaster.  But the chain is not all that likely to actually occur.}


  \item ``These environmentalists have gone overboard. Who are these crazies who dictate what we can do with our own property? This great nation was founded in the spirit of capitalism, and free enterprise! The sacred right of private property is grounded in the commandments of almighty God! Down with the nature-nuts! Away with the eco-maniacs!''
  
\textit{This commits the fallacy of appeal to the people, which is a fallacy of relevance.  The arguer evokes emotion to sway his listeners as opposed to actually addressing the environmentalists' arguments.}

  This is \textbf{not} an \textit{appeal to authority}.  While the arguer mentions God, one would think that God would constitute a qualified authority on just about any subject.
  \\
  
  \underline{Additionally receiving credit:} 
 \begin{description}
  \item [Full credit] Ad hominem abusive
  \item [Half credit] Red herring (if explanation given)
 \end{description}
  
  \item ``Libertarians are insensitive to the plight of the poor. This is made evident by their plan to abolish the minimum wage. They want to do this because they want to eliminate all restrictions on hiring the poor. And of course this is true because Libertarians just don't care about the poor.''
  
  \textit{This commits the fallacy of begging the question, which is a fallacy of presumption.  The first sentence is the conclusion, which is said to be supported by the following sentences.  But the final sentence is just a restatement of the first one.}
  
  This is \textbf{not} an \textit{ad hominem}.  The views of Libertarians are addressed directly, and their view on minimum wage seems relevant to the conclusion.  It is also \textbf{not} a hasty generalization.  The arguer does make a claim about Libertarians as a whole, but they don't do so on the basis of examining a few cases.
  
  \item ``Senators Harrison and Beaney are both republicans from states with small populations.  They're both known outdoorsmen, and they have records of sticking mostly by the party line.  Harrison supports the concealed weapon bill.  So, Beaney is probably in favor of it as well.''

\textit{This argument provides an analogy.  It seems to me to be a \textbf{strong} analogy on the basis that the similarities referenced are relevant to the conclusion.  But others have suggested to me that being an outdoorsman may not be that relevant to views on concealed weapons.  So I also accepted that this commits the fallacy of \textbf{weak analogy}.}
  
  %\item ``Max argues that we should upgrade our firefighting equipment by purchasing a fleet of new trucks with all the bells and whistles. But it’s only natural that he should argue this way. Max is a firefighter himself, and he just wants a fancy new truck to ride around in.''
  
  \item ``A spokesman for the government has argued that our insistence on a twelve percent increase in pay for steel workers is inflationary. But the government's own actions are far more inflationary than ours. The government just gave a fifteen percent increase to every single employee of the giant federal bureaucracy.''
  
\textit{This commits the fallacy of ad hominem tu quoque, which is a fallacy of relevance.  The arguer shifts from the issue of whether their policy is inflationary to the fact that the government has done the same thing.  Despite what the government has done, it is perfectly possible that the 12\% increase is still inflationary.}

This \textbf{does} commit a fallacy.  The conclusion is that the policy is not inflationary, but that conclusion is not supported by the premise given.  As the saying goes, two wrongs don't make a right.
   \\
   
  \underline{Additionally receiving credit:} 
 \begin{description}
  \item [Half credit] False dichotomy (if explanation given)
 \end{description}
 
  \item ``Clearly the moral standards of our youth have decayed. Two doctors at Central Hospital appeared recently on a talk show, and both were convinced that the youth of today have no morals at all.''
  
\textit{This commits the fallacy of appeal to unqualified authority, which is a fallacy of weak induction.  Doctors are experts in the field of medicine, but we aren't given any reason to think that their expertise extends to the nature of society's moral fabric.}
\\
  
  \underline{Additionally receiving credit:} 
 \begin{description}
  \item [Full credit] Hasty generalization
 \end{description}
  
  \item `` Most elderly people who are hospitalized with cancer eventually die from the disease. Thus, if an elderly person wants to recover from cancer he or she must, at all costs, refuse hospitalization.''
  
\textit{This commits the fallacy of false cause (common cause), which is a fallacy of weak induction.  The arguer reasons that since the death occurs after the hospitalization, the death is caused by the hospitalization.  But this causal relation does not exist.}

This is \textbf{not} a hasty generalization.  If it is true that \textit{most} elderly people with cancer die in the hosptial, then it is not hasty to generalize from that.

\end{enumerate}  

\vspace{1in}
  
\paragraph{Bonus:}  The fallacies of \textbf{strawman} and \textbf{red herring} are both fallacies of relevance, but they involve slightly different shifts in the argument.  What is the main difference between these two types of fallacy? (\textbf{2 points})

\textit{When one commits a straw man, they attack a weaker argument for the same conclusion, while a red herring is an attack on an argument for a different conclusion altogether.}

\end{document}
