\documentclass[10pt,landscape]{article}

\usepackage[letterpaper]{geometry}
\usepackage{enumerate,verbatim,mdwlist}
\usepackage{fancyhdr}
\usepackage{linguex}
\usepackage[colorlinks=true,linkcolor=blue]{hyperref}
\usepackage{multicol}

\setlength{\columnseprule}{.5pt}

%Packages needed for trees
\usepackage{amsfonts,amsmath,amssymb}
\usepackage[varg]{txfonts}
\usepackage{qtree}

%%%%%%%%%%%%%%%%%%%%%%%%%%%%%%%%%%%%%%%%%%%%%
%% Represent categorical logic graphically %%
%%%%%%%%%%%%%%%%%%%%%%%%%%%%%%%%%%%%%%%%%%%%%

\usepackage{tikz}
\usepackage{xstring}

\usetikzlibrary{shapes,backgrounds}
\tikzstyle{every node}=[font=\tiny] 

%%%%%%%%%%%%%%%%%%%%%%%%%%%%%%%%
%% Draw squares of Opposition %%
%%%%%%%%%%%%%%%%%%%%%%%%%%%%%%%%

\def\oppsquare{(-1.5,1) node[above left]{$All$} -- (1.5,1) node[above right]{$No$} -- (1.5,-1) node[below right]{$Some-not$} -- (-1.5,-1) node[below left]{$Some$} -- (-1.5,1)}
\def\oppcross{(-1.5,1) -- (1.5,-1) node[sloped,above,pos=0.3]{contra} node[sloped,above,pos=0.7]{dictory} (-1.5,-1) -- (1.5,1) node[sloped,above,pos=0.3]{contra} node[sloped,above,pos=0.7]{dictory}}

\def\sqroppMod{%
  \begin{scope}
    \draw \oppsquare;
    \draw \oppcross;
    \draw (-1.75,0) node[rotate=90] {undet.};
    \draw (0,1.25) node {undet.};
    \draw (0,-1.25) node {undet.};
    \draw (1.75,0) node[rotate=90] {undet.};
  \end{scope}
}

\def\sqroppTrad{%
  \begin{scope}
    \draw \oppsquare;
    \draw \oppcross;
    \draw (-1.75,0) node[rotate=90] {subalt.};
    \draw (0,1.25) node {contrary};
    \draw (0,-1.25) node {subcontrary};
    \draw (1.75,0) node[rotate=90] {subalt.};
  \end{scope}
}

%%%%%%%%%%% Turn categorical props into Venn diagrams %%%%%%%%%%%%%%%
%% \catvenn{title:<y/n>}{quantifier:<All/No/Some>}{quality:<not/>} %%
%%         {complement:<non-/>}{subject class}                     %%
%%         {complement:<non-/>}{predicate class}                   %%
%%%%%%%%%%%%%%%%%%%%%%%%%%%%%%%%%%%%%%%%%%%%%%%%%%%%%%%%%%%%%%%%%%%%%

%%%%%%%%%%%%%
%% Circles %%
%%%%%%%%%%%%%

%% Venn circles
\def\firstcircle{(0,0) circle (1cm)} 
\def\secondcircle{(0:1.5cm) circle (1cm)}
\def\thirdcircle{(60:1.5cm) circle (1cm)}

%% Overlapping and disjoint circles
\def\firstcircleN{(0,0) circle (.75cm) node [below left=.25in] {$A$}}
\def\secondcircleN{(0:2cm) circle (.75cm) node [below right=.25in] {$B$}}
\def\firstcircleA{(0,0) circle (1.25cm) node [above right] {$A$}}
\def\secondcircleA{(.125,0) circle (.75cm) node [below left=.25in] {$B$}}

%%%%%%%%%%%%%%%%%%%%%%%%%%%
%% Venn diagram template %%
%%%%%%%%%%%%%%%%%%%%%%%%%%%

\newcommand{\vennbox}[2]{%
  \draw (-1.5,-1.5) rectangle (3,1.5);
  \draw \firstcircle node [below left=.25in] {#1};
  \draw \secondcircle node [below right=.25in] {#2};
}

\newcommand{\syllbox}[3]{%
  \draw (-1.5,-1.5) rectangle (3,2.75);
  \draw \firstcircle node [below left=.25in] {#1};
  \draw \secondcircle node [below right=.25in] {#2};
  \draw \thirdcircle node [above right=.25in] {#3};
}

%%%%%%%%%%%%%%%%%%%%
%% Universal Defs %%
%%%%%%%%%%%%%%%%%%%%

\def\fillleft{%
  \begin{scope}[even odd rule, fill opacity=0.5]
    \clip \secondcircle (-1,-1) rectangle (1,1);
    \fill[blue] \firstcircle;
  \end{scope}
}

\def\fillmiddle{%
  \begin{scope}[fill opacity=0.5]
    \clip \firstcircle;
    \fill[blue] \secondcircle;
  \end{scope}
}

\def\fillright{%
  \begin{scope}[even odd rule, fill opacity=0.5]
    \clip \firstcircle (-1,-1) rectangle (2.5,1);
    \fill[blue] \secondcircle;
  \end{scope}
}

\def\fillbox{%
  \begin{scope}
    \fill[fill opacity=0.5, blue] (-1.5,-1.5) rectangle (3,1.5);
    \fill[fill opacity=1, white] \firstcircle \secondcircle;
  \end{scope}
}

%%%%%%%%%%%%%%%%%%%%%%
%% Existential Defs %%
%%%%%%%%%%%%%%%%%%%%%%

\def\xleft{%
  \node {x};
}
\def\xleftA{%
\draw (0,0) node [draw,rounded corners] {x};
}%

\def\xmiddle{%
  \node [right=.6cm] {x};
}
\def\xmidA{%
  \draw (.75cm,0) node [draw,rounded corners] {x};
}%

\def\xright{%
  \node [right=1.5cm] {x};
}

\def\xbox{%
 \node [below right=1.1cm] {x};
}

%%%%%%%%%%%%%%%%%%%
%% Draw diagrams %%
%%%%%%%%%%%%%%%%%%%

\newcommand{\catvenn}[7]{%
  \IfEqCase{#2}{%
    {All}{%
      \IfEqCase{#4}{%
	{non-}{%
	  \IfEqCase{#6}{%
	    {non-}{\fillright}%
	    {}{\fillbox}%
	  }[\PackageError{catvenn}{Undefined option to tree: pred-non}{}]%
	}%
	{}{%
	  \IfEqCase{#6}{%
	    {non-}{\fillmiddle}%
	    {}{\fillleft}%
	  }[\PackageError{catvenn}{Undefined option to tree: pred-non}{}]%
	}%
      }[\PackageError{catvenn}{Undefined option to tree: subj-non}{}]%
    }%
    {No}{%
      \IfEqCase{#4}{%
	{non-}{%
	  \IfEqCase{#6}{%
	    {non-}{\fillbox}%
	    {}{\fillright}%
	  }[\PackageError{catvenn}{Undefined option to tree: pred-non}{}]%
	}%
	{}{%
	  \IfEqCase{#6}{%
	    {non-}{\fillleft}%
	    {}{\fillmiddle}%
	  }[\PackageError{catvenn}{Undefined option to tree: pred-non}{}]%
	}%
      }[\PackageError{catvenn}{Undefined option to tree: subj-non}{}]%
    }%
    {Some}{%
      \IfEqCase{#3}{%
	{not}{%
	  \IfEqCase{#4}{%
	    {non-}{%
	      \IfEqCase{#6}{%
		{non-}{\xright}%
		{}{\xbox}%
	      }[\PackageError{catvenn}{Undefined option to tree: pred-non}{}]%
	    }%
	    {}{%
	      \IfEqCase{#6}{%
		{non-}{\xmiddle}%
		{}{\xleft}%
	      }[\PackageError{catvenn}{Undefined option to tree: pred-non}{}]%
	    }%
	  }[\PackageError{catvenn}{Undefined option to tree: subj-non}{}]%
	}%
	{}{%
	  \IfEqCase{#4}{%
	    {non-}{%
	      \IfEqCase{#6}{%
		{non-}{\xbox}%
		{}{\xright}%
	      }[\PackageError{catvenn}{Undefined option to tree: pred-non}{}]%
	    }%
	    {}{%
	      \IfEqCase{#6}{%
		{non-}{\xleft}%
		{}{\xmiddle}%
	      }[\PackageError{catvenn}{Undefined option to tree: pred-non}{}]%
	    }%
	  }[\PackageError{catvenn}{Undefined option to tree: subj-non}{}]%
	}%
      }[\PackageError{catvenn}{Undefined option to tree: not}{}]%
    }%
  }[\PackageError{catvenn}{Undefined option to tree: quant}{}]%
  
  \IfEqCase{#1}{%
    {y}{\draw (0,1.25) node {#2 #4#5 are #3 #6#7};}%
    {n}{}%
  }[\PackageError{catvenn}{Undefined option to tree: title}{}]%
  
  \vennbox{#5}{#7}%
}%

%%%%%%%%%%%%%%%%%%%%%%%%%%%%
%% Categorical syllogisms %%
%%%%%%%%%%%%%%%%%%%%%%%%%%%%

\newcommand{\filltopleft}[1]{%
  \begin{scope}[fill opacity=0.5]
    \clip \firstcircle;
    \fill[#1] \thirdcircle;
  \end{scope}
}

\newcommand{\filltopright}[1]{%
  \begin{scope}[fill opacity=0.5]
    \clip \secondcircle;
    \fill[#1] \thirdcircle;
  \end{scope}
}

\def\filltopfirst{%
  \begin{scope}[even odd rule, fill opacity=0.5]
    \clip \firstcircle (-1.5,-1.5) rectangle (3,2.75);
    \fill[green] \thirdcircle;
  \end{scope}
}

\def\filltopsecond{%
  \begin{scope}[even odd rule, fill opacity=0.5]
    \clip \secondcircle (-1.5,-1.5) rectangle (3,2.75);
    \fill[green] \thirdcircle;
  \end{scope}
}

\def\xtopleft{%
  \draw (60:.75cm) node [text=red] {x};
}

\def\xtopright{%
  \draw (30:1.3cm) node {x};
}

\def\xtopmid{%
  \draw (30:.85cm) node [text=red] {X};
}



%% Margin Setting
\geometry{hmargin={.5in,.5in},vmargin={1in,1in}}
\setlength{\parindent}{0.0in}
%\setlength{\parskip}{3mm}
\setlength{\tabcolsep}{10pt}
\setlength{\arraycolsep}{10pt}

%% Header
\setlength{\headheight}{15pt}
\pagestyle{fancy}
\fancyhead{}
\fancyhead[L]{Phil 101, f14}
\fancyhead[C]{Midterm review sheet}
\fancyhead[R]{Exam date: October 20, 2014}

\begin{document}

\paragraph{Preparing for the midterm exam:} Below is an exhaustive list of the topics covered since the beginning of class.  Use this list to help organize your studying for the midterm.

\vspace{3mm}

You will be allowed to bring \textbf{one piece of paper} to the exam.  You can write whatever you want on this piece of paper.  Use it to jot down hard to remember distinctions or definitions.  You will be required to turn in your notes sheet with your completed exam.

\hrulefill

\begin{multicols}{2}
 
\paragraph{Reasons}
  \begin{enumerate}
   \item Epistemic v. pragmatic
   \item Justificatory v. explanatory
   \item Subjective v. objective
   \item First blush v. all things considered
  \suspend{enumerate}
  
\paragraph{Arguments}
  \resume{enumerate}
   \item Structure
    \begin{enumerate}
     \item Premises
     \item Conclusion
     \item Support relation
    \end{enumerate}
   \item Kinds
    \begin{enumerate}
     \item Deductive
     \item Inductive
    \end{enumerate}
   \item Evaluation
    \begin{enumerate}
     \item Validity and soundness
     \item Strength and cogency
    \end{enumerate}
  \suspend{enumerate}
  
\paragraph{Propositions}
  \resume{enumerate}
   \item Atomic propositions
    \begin{enumerate}
     \item Subject and predicate
     \item Particular
     \item General
    \end{enumerate}
   \item Compound propositions
    \begin{enumerate}
     \item Negation
     \item Disjunction
     \item Conjunction
     \item Conditional
      \begin{enumerate}
       \item Antecedent
       \item Consequent
      \end{enumerate}
    \end{enumerate}
  \item Content v. force
  \suspend{enumerate}
\paragraph{Word meanings}

  \resume{enumerate}
   \item Cognitive v. emotive meaning
   \item Vagueness
   \item Ambiguity
   \item Intension v. extension
  \suspend{enumerate}
  
\paragraph{Informal fallacies}
  \resume{enumerate}
    \item What does \textit{non sequiter} mean?
    \item Fallacies of relevance
      \begin{enumerate}
       \item Appeals to force, pity, or the people
       \item Argument against the person (ad hominem)
	\begin{enumerate}
	 \item Abusive
	 \item Circumstantial
	 \item Tu quoque
	\end{enumerate}
       \item Strawman
       \item Red herring
      \end{enumerate}
    \item Fallacies of weak induction
      \begin{enumerate}
       \item Appeal to unqualified authority
       \item Appeal to ignorance
       \item Hasty generalization
       \item False cause
	\begin{enumerate}
	 \item \textit{Post hoc ergo propter hoc}
	 \item \textit{Non causa pro causa}
	 \item \textit{Common cause}
	 \item \textit{Gambler's fallacy}
	\end{enumerate}
       \item Slippery slope
       \item Weak analogy
      \end{enumerate}
   \item Fallacies of presumption
      \begin{enumerate}
        \item Begging the question
        \item Complex question
        \item False dichotomy
      \end{enumerate}
    \item Fallacies of ambiguity
      \begin{enumerate}
        \item Equivocation
      \end{enumerate}
    \item Fallacies of illicit transference
      \begin{enumerate}
	\item Composition
	\item Division
      \end{enumerate}
  \suspend{enumerate}
  
\paragraph{The language of categorical logic}
  \resume{enumerate}
    \item Vocabulary
      \begin{enumerate}
       \item Quantifiers
	 \begin{enumerate}
	  \item Quantity
	    \begin{enumerate}
	     \item Universal
	     \item Existential
	    \end{enumerate}
	  \item Quality
	    \begin{enumerate}
	     \item Affirmative
	     \item Negative
	    \end{enumerate}
	 \end{enumerate}
       \item Subject and predicate terms
       \item Copula
       \item Complement classes
      \end{enumerate}
   \item Syntax
      \begin{enumerate}
       \item Standard form categorical propositions
      \end{enumerate}
   \item Semantics
      \begin{enumerate}
       \item Classes of individuals
       \item Inclusion and exclusion
       \item Existential import
	  \begin{enumerate}
	   \item Boolean perspective
	   \item Aristotelian perspective
	  \end{enumerate}
      \end{enumerate}
\suspend{enumerate}

\paragraph{Relations between categorical propositions}
  \resume{enumerate}
    \item Syntactic relations
      \begin{enumerate}
       \item Conversion
       \item Obversion
       \item Contraposition
       \item Which relations preserve truth?
      \end{enumerate}
    \item Semantic relations
      \begin{enumerate}
       \item Contradiction
       \item Equivalence
       \item Compatibility
       \item Contrary (aristotelian only)
       \item Subcontrary (aristotelian only)
       \item Subalternation (aristotelian only)
      \end{enumerate}
  \suspend{enumerate}

\paragraph{Representing categorical propositions}
  \resume{enumerate}
    \item Squares of opposition
      \begin{enumerate}
       \item Modern square (boolean)
       \item Traditional square (aristotelian)
       \item Which propositions stand in which semantic relations?
      \end{enumerate}
    \item Venn diagrams
      \begin{enumerate}
       \item Circles represent classes
       \item Shading represents emptiness
       \item X's represent existence
      \end{enumerate}
  \suspend{enumerate}
  
\paragraph{Arguments in categorical logic}
  \resume{enumerate}
    \item Evaluating arguments
    \begin{enumerate}
    \item Validity: premises say at least as much as the conclusion
    \item Invalidity
      \begin{enumerate}
       \item Conclusion says more than the premises
       \item Conclusion conflicts with the premises
      \end{enumerate}
    \item Conditional validity: argument is valid, but contains universal propositions
    \end{enumerate}
    \item Direct inferences: one premise arguments.
      \begin{enumerate}
       \item Evaluation method \#1:
	\begin{enumerate}
	 \item Translate all propositions into standard form (using contradictory relation)
	 \item Construct and compare Venn diagrams
	\end{enumerate}
      \item Evaluation method \#2: Use semantic relations on the square of opposition to infer the truth value of the conclusion from the truth value of the premise.
      \end{enumerate}
    \suspend{enumerate}
      
\paragraph{Categorical syllogisms}
   \resume{enumerate}
    \item Structure
      \begin{enumerate}
       \item Major premise (major term and middle term)
       \item Minor premise (minor term and middle term)
       \item Conclusion (subject = minor term, predicate = major term)
       \item Standard form (important to apply rules of formal logic)
      \end{enumerate}

       \item Mood
	\begin{enumerate}
	 \item Derived from the kinds of proposition in the argument
	 \item A = All, E = No, I = Some, O = Some-not
	 \item 64 possible syllogisms
	\end{enumerate}
      \item Figure
	\begin{enumerate}
	 \item Derived from relative position of terms in the propositions
	 \item 4 possible figures
	\end{enumerate}
      \item Validity tables: look up syllogism by mood and figure to see if it's valid
      \item Venn diagrams
	\begin{enumerate}
	 \item Valid syllogisms have all three propositions true together
	 \item Use a single diagram with three circles (one each for major term, minor term, middle term)
	 \item Draw each premise into the diagram
	  \begin{enumerate}
	   \item Start with universal propositions
	   \item Be sure to shade the entire relevant area.
	   \item If an X could go in two different areas, put it on the line between them
	  \end{enumerate}
	\item Look to see if the conclusion is represented in the diagram
	  \begin{enumerate}
	   \item If it is, then the premises say what the conclusion says, and the argument is valid.
	   \item If it isn't, then the argument is invalid.
	  \end{enumerate}
	\end{enumerate}
      \end{enumerate}

\end{multicols}

\end{document}
