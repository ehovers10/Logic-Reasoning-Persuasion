\documentclass[10pt]{article}

\usepackage[letterpaper]{geometry}
\usepackage{enumerate,verbatim,mdwlist}
\usepackage{fancyhdr}
\usepackage{linguex}
\usepackage[colorlinks=true,linkcolor=blue]{hyperref}

%Packages needed for trees
\usepackage{amsfonts,amsmath,amssymb}
\usepackage[varg]{txfonts}
\usepackage{qtree}

%% Margin Setting
\geometry{hmargin={.5in,.5in},vmargin={1in,1in}}
\setlength{\parindent}{0.0in}
\setlength{\parskip}{3mm}
\setlength{\tabcolsep}{10pt}
\setlength{\arraycolsep}{10pt}

%% Header
\setlength{\headheight}{23pt}
\pagestyle{fancy}
\fancyhead{}
\fancyhead[L]{Phil 101, f14}
\fancyhead[C]{Homework \# 1}
\fancyhead[R]{Due: Monday September 22, 2014 \\ Point total: 20 points}

\begin{document}

\textbf{Name:}\underline{\hspace{2in}}



\begin{enumerate}
  \item Does the following passage present an \textit{epistemic} or a \textit{pragmatic} reason for belief? (\textbf{1 point}) Briefly explain your answer. (\textbf{1 point})
  
  \begin{quotation}
    God is, or He is not. But to which side shall we incline? Let us see. Since you must choose, let us see which interests you least. You have two things to lose, the true and the good; and two things to stake, your reason and your will, your knowledge and your happiness; and your nature has two things to shun, error and misery. Your reason is no more shocked in choosing one rather than the other, since you must of necessity choose... But your happiness? Let us weigh the gain and the loss in wagering that God is... If you gain, you gain all; if you lose, you lose nothing. Wager, then, without hesitation that He is. (Blaise Pascal, \textit{Pense\`es})
  \end{quotation}
  
   \begin{tabular}{ll}
    Kind of reason: & \underline{\hspace{1.5in}} \\
    Explanation: & \\
  \end{tabular}
  
\suspend{enumerate}

\vspace{.7in}
  
For each of the following passages, extract the argument and put it into premise/conclusion form (\textbf{3 points}). Don't forget to include any implicit premises that may be crucial to the argument. Then, identify whether the argument you extracted is intended to be deductive or inductive (\textbf{1 point}).  Finally, assess the argument (\textbf{1 point}): if it is deductive, is it valid?  if it is inductive, is it strong? Briefly explain why you give that assessment (\textbf{1 point}).
 
\resume{enumerate} 
   \item If it rains all day today, Phil will spend it binge watching Game of Thrones. But it's not going to rain all day.  So, Phil won't spend the day binge watching Game of Thrones.
  
  \begin{enumerate}[1.]
    \item \underline{\hspace{6in}}
    \item \underline{\hspace{6in}}
    \item \underline{$\therefore,$\hspace{5.9in}}
  \end{enumerate}
  
  \begin{tabular}{ll}
    Kind of argument: & \underline{\hspace{1.5in}} \\
    Assessment: & \underline{\hspace{1.5in}} \\
    Explanation: & \\
  \end{tabular}

\vspace{.9in}

  \item New discoveries suggest that many dinosaurs probably had feathers.  Recently uncovered fossils show apparent patterns in the skin that are very similar to patterns found in the skin of modern day birds.
  
    \begin{enumerate}[1.]
    \item \underline{\hspace{6in}}
    \item \underline{\hspace{6in}}
    \item \underline{$\therefore,$\hspace{5.9in}}
  \end{enumerate}
  
  \begin{tabular}{ll}
    Kind of argument: & \underline{\hspace{1.5in}} \\
    Assessment: & \underline{\hspace{1.5in}} \\
    Explanation: & \\
  \end{tabular}

\vspace{.9in}
  
  \item I'm certain that Clyde is trying to steal my best friend.  After all, Bertha says she saw them at the movies together.  And, yesterday, my friend got a text that he wouldn't show me.  It had to be from Clyde.

  \begin{enumerate}[1.]
    \item \underline{\hspace{6in}}
    \item \underline{\hspace{6in}}
    \item \underline{\hspace{6in}}
    \item \underline{$\therefore,$\hspace{5.9in}}
  \end{enumerate}
  
  \begin{tabular}{ll}
    Kind of argument: & \underline{\hspace{1.5in}} \\
    Assessment: & \underline{\hspace{1.5in}} \\
    Explanation: & \\
  \end{tabular}
  
\vspace{1in}  
  
  \item \textbf{Bonus:} What is the definition of \textit{validity} that we are using in this class? (\textbf{1 point}) Explain how the \textit{method of counterexample} is used to prove that an argument is \textit{invalid}. (\textbf{2 points})

\end{enumerate}
\end{document}
