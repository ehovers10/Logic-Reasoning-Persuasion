\documentclass[10pt]{article}

\usepackage[letterpaper]{geometry}
\usepackage{enumerate,verbatim,mdwlist}
\usepackage{fancyhdr}
\usepackage{linguex}
\usepackage[colorlinks=true,linkcolor=blue]{hyperref}

%Packages needed for trees
\usepackage{amsfonts,amsmath,amssymb}
\usepackage[varg]{txfonts}
\usepackage{qtree}

%% Margin Setting
\geometry{hmargin={.5in,.5in},vmargin={1in,1in}}
\setlength{\parindent}{0.0in}
\setlength{\parskip}{3mm}
\setlength{\tabcolsep}{10pt}
\setlength{\arraycolsep}{10pt}

%% Header
\setlength{\headheight}{23pt}
\pagestyle{fancy}
\fancyhead{}
\fancyhead[L]{Phil 101, f14}
\fancyhead[C]{Homework \# 1}
\fancyhead[R]{Due: Monday September 22, 2014 \\ Point total: 20 points}

\begin{document}

\textbf{Name:}\underline{Answer Key}



\begin{enumerate}
  \item Does the following passage present an \textit{epistemic} or a \textit{pragmatic} reason for belief? (\textbf{1 point}) Briefly explain your answer. (\textbf{1 point})
  
  \begin{quotation}
    God is, or He is not. But to which side shall we incline? Let us see. Since you must choose, let us see which interests you least. You have two things to lose, the true and the good; and two things to stake, your reason and your will, your knowledge and your happiness; and your nature has two things to shun, error and misery. Your reason is no more shocked in choosing one rather than the other, since you must of necessity choose... But your happiness? Let us weigh the gain and the loss in wagering that God is... If you gain, you gain all; if you lose, you lose nothing. Wager, then, without hesitation that He is. (Blaise Pascal, \textit{Pense\`es})
  \end{quotation}
  
   \begin{tabular}{ll}
    Kind of reason: & \underline{Pragmatic reason} \\
    Explanation: & \\
  \end{tabular}
  
  \textit{Pragmatic reasons have to do with the benefits of believing as opposed to the truth of the proposition. Pascal doesn't say that God definitely exists; he is saying that you stand to benefit with potential eternal salvation if you believe in God, so you ought to do it.}
  
\suspend{enumerate}

\vspace{.7in}
  
For each of the following passages, extract the argument and put it into premise/conclusion form (\textbf{3 points}). Don't forget to include any implicit premises that may be crucial to the argument. Then, identify whether the argument you extracted is intended to be deductive or inductive (\textbf{1 point}).  Finally, assess the argument (\textbf{1 point}): if it is deductive, is it valid?  if it is inductive, is it strong? Briefly explain why you give that assessment (\textbf{1 point}).
 
\resume{enumerate} 
   \item If it rains all day today, Phil will spend it binge watching Game of Thrones. But it's not going to rain all day.  So, Phil won't spend the day binge watching Game of Thrones.
  
  \begin{enumerate}[1.]
    \item \underline{If it rains all day, Phill will spend it watching GoT.}
    \item \underline{It won't rain all day.}
    \item \underline{$\therefore,$ Phil won't spend the day watching GoT.}
  \end{enumerate}
  
  \begin{tabular}{ll}
    Kind of argument: & \underline{Deductive} \\
    Assessment: & \underline{Invalid} \\
    Explanation: &  \\
  \end{tabular}
  
  \textit{Imagine that Phil has a cold and can barely get out of bed.  In this scenario, Phil will watch GoT all day whether it rains or not.  So, both premises can be true, but the conclusion is still false.}

\vspace{.9in}

  \item New discoveries suggest that many dinosaurs probably had feathers.  Recently uncovered fossils show apparent patterns in the skin that are very similar to patterns found in the skin of modern day birds.
  
    \begin{enumerate}[1.]
    \item \underline{Recently uncovered fossils show patterns in the skin of dinosaurs that are similar to those in birds.}
    \item \underline{Birds have feathers.}
    \item \underline{$\therefore,$ Many dinosaurs probably had feathers.}
  \end{enumerate}
  
  \begin{tabular}{ll}
    Kind of argument: & \underline{Inductive} \\
    Assessment: & \underline{Strong} \\
    Explanation: &  \\
  \end{tabular}
  
  \textit{The argument provides an analogy between birds and dinosaurs.  Since feathers are the only things we know of that create patterns like that, this seems like a good analogy.}

\vspace{.9in}
  
  \item I'm certain that Clyde is trying to steal my best friend.  After all, Bertha says she saw them at the movies together.  And, yesterday, my friend got a text that he wouldn't show me.  It had to be from Clyde.

  \begin{enumerate}[1.]
    \item \underline{Bertha saw Clyde and my friend at the movies.}
    \item \underline{My friend got a text from Clyde that he wouldn't show me.}
    \item \underline{Going to the movies with someone and sending them texts are friend stealing behaviors.}
    \item \underline{$\therefore,$ Clyde is trying to steal my best friend.}
  \end{enumerate}
  
  \begin{tabular}{ll}
    Kind of argument: & \underline{Deductive} \\
    Assessment: & \underline{Valid} \\
    Explanation: &  \\
  \end{tabular}
  
  \textit{On this reconstruction, the conclusion does follow from the premises.  If all the premises are true, the conclusion must be.  But premise \#3 doesn't seem true.  One can do those things without actually hoping to steal anyone's friend.}
  
\vspace{1in}  
  
  \item \textbf{Bonus:} What is the definition of \textit{validity} that we are using in this class? (\textbf{1 point}) Explain how the \textit{method of counterexample} is used to prove that an argument is \textit{invalid}. (\textbf{2 points})
  
  \textit{A deductive argument is valid just in case, if the premises are all true, then the conclusion must be true.  The method of counterexample shows that there is a coherent situation in which the premises can be true without the conclusion also being true.  This is enough to show that true premises don't guarantee a true conclusion, so the argument is invalid.}

\end{enumerate}
\end{document}
