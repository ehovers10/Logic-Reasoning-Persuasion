\documentclass[10pt]{article}

\usepackage[letterpaper]{geometry}
\usepackage{enumerate,verbatim,mdwlist}
\usepackage{fancyhdr}
\usepackage{linguex}
\usepackage[colorlinks=true,linkcolor=blue]{hyperref}

%Packages needed for trees
\usepackage{amsfonts,amsmath,amssymb}
\usepackage[varg]{txfonts}
\usepackage{qtree}

\include{tikz-venn-diagrams}

%% Margin Setting
\geometry{hmargin={.5in,.5in},vmargin={1in,1in}}
\setlength{\parindent}{0.0in}
%\setlength{\parskip}{2mm}
\setlength{\tabcolsep}{10pt}
\setlength{\arraycolsep}{10pt}

%% Header
\setlength{\headheight}{23pt}
\pagestyle{fancy}
\fancyhead{}
\fancyhead[L]{Phil 101, f14}
\fancyhead[C]{Homework \# 3}
\fancyhead[R]{Due: Thursday October 16, 2014 \\ Point total: 20 points}

\begin{document}

\small

\textbf{Name:}\underline{\hspace{2in}}

\paragraph{The language of categorical logic}

\begin{enumerate}
 \item What is the primary difference between the \textbf{Boolean} and \textbf{Aristotelian} perspectives on categorical logic? \textbf{(1 point)}
 
\suspend{enumerate}

\vspace{1.5cm}

\paragraph{Syntactic relations}

\resume{enumerate}
  \item Write the converse of \textit{Some A are non-B}: \underline{\hspace{3in}} \textbf{(1 point)}
  
  \item Write the obverse of \textit{No Q are S}: \underline{\hspace{3in}} \textbf{(1 point)}
  
  \item Write the contraposition of \textit{Some non-M are not N}: \underline{\hspace{3in}} \textbf{(1 point)}
\suspend{enumerate}

\paragraph{Semantic relations}

\resume{enumerate}
  \item Assume that \textit{All A are B} is true. 
  
  Then the truth value of \textit{No A are B} is: \hspace{1cm} True \hspace{1cm} False \hspace{1cm} Undetermined \hspace{1cm} \textbf{(circle one, 1 point)} \\
  
  The relation that allows us to make that claim is: \underline{\hspace{3in}} \textbf{(1 point)}
  
  \item Assume that \textit{Some F are G} is false. 
  
  Then the truth value of \textit{All F are G} is: \hspace{1cm} True \hspace{1cm} False \hspace{1cm} Undetermined \hspace{1cm} \textbf{(circle one, 1 point)} \\
  
  The relation that allows us to make that claim is: \underline{\hspace{3in}} \textbf{(1 point)}
  
\suspend{enumerate}

\paragraph{Direct inferences}

\resume{enumerate}
  \item Assume the \textbf{Boolean perspective}, and consider the direct inference: \textit{Some W are not X}. Therefore, it is false that \textit{no W are X}.
  
  \vspace{3mm}
  
  This inference is: \hspace{1cm} Valid \hspace{1cm} Invalid \hspace{1cm} Conditionally valid \hspace{1cm} \textbf{(circle one, 1 point)}
  
  \vspace{3mm}
  
  Justify your answer in the space below.  You may use either method for assessing the validity of direct inferences. \textbf{(2 points)}
  
  \vspace{4cm}
  
  \item Assume the \textbf{Aristotelian perspective}, and consider the direct inference: It is false that \textit{some R are S}. Therefore, it is false that \textit{All R are S}.
  
  \vspace{3mm}
  
  This inference is: \hspace{1cm} Valid \hspace{1cm} Invalid \hspace{1cm} Conditionally valid \hspace{1cm} \textbf{(circle one, 1 point)}
  
  \vspace{3mm}
  
  Justify your answer in the space below.  You may use either method for assessing the validity of direct inferences. \textbf{(2 points)}
  
\suspend{enumerate}

\newpage

\paragraph{Syllogisms}

\resume{enumerate}
  \item Consider the syllogism:
    \begin{enumerate}[1)]
     \item All P are M
     \item All M are S
     \item $\therefore,$ some S are P
    \end{enumerate}

    This syllogism is: \hspace{1cm} Valid \hspace{1cm} Invalid \hspace{1cm} Conditionally valid \hspace{1cm} \textbf{(circle one, 1 point)}
    
    \vspace{3mm}
    
    Justify your answer using the Venn diagram below \textbf{(2 points)}
    
    \begin{center}
    \begin{tikzpicture}
     \syllbox{S}{P}{M}
    \end{tikzpicture}
    \end{center}

  \item Consider the syllogism:
    \begin{enumerate}[1)]
     \item Some M are P
     \item No S are M
     \item $\therefore,$ some S are not P
    \end{enumerate}

    This syllogism is: \hspace{1cm} Valid \hspace{1cm} Invalid \hspace{1cm} Conditionally valid \hspace{1cm} \textbf{(circle one, 1 point)}
    
    \vspace{3mm}
    
    Justify your answer using the Venn diagram below \textbf{(2 points)}
    
    \begin{center}
    \begin{tikzpicture}
     \syllbox{S}{P}{M}
    \end{tikzpicture}
    \end{center}
  
\end{enumerate}
  
\paragraph{Bonus:}  Demonstrate the fact that \textit{All Y are Z} and its obverse always have the same truth value. \textbf{(3 points)}

\end{document}
