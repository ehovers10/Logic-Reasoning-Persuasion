\documentclass[10pt]{article}

\usepackage[letterpaper]{geometry}
\usepackage{enumerate,verbatim,mdwlist}
\usepackage{fancyhdr}
\usepackage{linguex}
\usepackage[colorlinks=true,linkcolor=blue]{hyperref}

%Packages needed for trees
\usepackage{amsfonts,amsmath,amssymb}
\usepackage[varg]{txfonts}
\usepackage{qtree}
\usepackage{multicol}

%% Margin Setting
\geometry{hmargin={.5in,.5in},vmargin={1in,1in}}
\setlength{\parindent}{0.0in}
%\setlength{\parskip}{2mm}
\setlength{\tabcolsep}{10pt}
\setlength{\arraycolsep}{10pt}

%% Header
\setlength{\headheight}{20.5pt}
\pagestyle{fancy}
\fancyhead{}
\fancyhead[L]{\small Phil 101, f14}
\fancyhead[C]{\small Bonus on natural deduction}
\fancyhead[R]{\small Due: Monday November 10, 2014 \\ Points possible: 10}
\fancyfoot{}

\begin{document}

\small

\textbf{Name:}\underline{\hspace{2in}}

\paragraph{Directions:} Construct complete proofs for the arguments below. Proofs from \underline{Group 1} are worth \textbf{3 points} and should take you about 5 lines to complete. Proofs from \underline{Group 2} are worth \textbf{3 points} and should take you about 6-7 lines to complete.  Proofs from \underline{Group 3} are worth \textbf{4 points} and will take about 11 lines to complete. You may submit up to 3 proofs for bonus credit.

\paragraph{Group 1}

\begin{enumerate}
  \item \[\begin{array}{lll}
         1. & F \supset G & \\
         2. & B \supset ( B \supset J ) & / F \supset J \\
        \end{array}\]
  \item \[\begin{array}{lll}
         1. & M \supset ( U \supset H ) & \\
         2. & (H \vee \sim U) \supset F & / M \supset F \\
        \end{array}\]

\suspend{enumerate}

\paragraph{Group 2} 

\resume{enumerate}
  \item \[\begin{array}{lll}
         1. & F \supset G & \\
         2. & B \supset ( B \supset J ) & / F \supset J \\
        \end{array}\]
  \item \[\begin{array}{lll}
         1. & M \supset ( U \supset H ) & \\
         2. & (H \vee \sim U) \supset F & / M \supset F \\
        \end{array}\]
\suspend{enumerate}


\paragraph{Group 3} 

\resume{enumerate}
  \item \[\begin{array}{lll}
         1. & F \supset G & \\
         2. & B \supset ( B \supset J ) & / F \supset J \\
        \end{array}\]
  \item \[\begin{array}{lll}
         1. & M \supset ( U \supset H ) & \\
         2. & (H \vee \sim U) \supset F & / M \supset F \\
        \end{array}\]
\end{enumerate}

\end{document}
