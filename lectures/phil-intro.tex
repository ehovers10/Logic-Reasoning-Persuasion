\documentclass[10pt,letterpaper,xcolor=dvipsnames,handout]{beamer}

%\usepackage[colorlinks=true,linkcolor=blue]{hyperref}
\usepackage{amssymb,mathabx}
\usepackage{linguex}
\usepackage{verbatim,enumerate,multirow}
\usepackage{xcolor}
%\usepackage{floatflt}

\usepackage{pgfpages} % to put several slides on one page
\pgfpagesuselayout{4 on 1}[letterpaper, landscape, border shrink=5mm]

\mode<article>{}

%Template themes
\usetheme{boxes}
%\useoutertheme{miniframes}
\useoutertheme{shadow}

%Templates
\setbeamertemplate{blocks}[rounded][shadow=true]
\setbeamertemplate{navigation symbols}[vertical]
\setbeamertemplate{section in head/foot shaded}[default][20]
\setbeamertemplate{title}

\setbeamertemplate{headline}
{
	\begin{beamercolorbox}[ht=3ex,dp=1ex]{erikcolor1}
		\insertshorttitle
		%\insertsectionnavigationhorizontal{\textwidth}{}{}
		%\usebeamerfont{title in head/foot}
	\end{beamercolorbox}
	\begin{beamercolorbox}[ht=3ex,dp=2ex]{erikcolor2}
	  \insertsectionnavigationhorizontal{\textwidth}{}{}
		%\insertsubsectionnavigationhorizontal{\textwidth}{}{}
		%\insertsubsection
	\end{beamercolorbox}
}
\setbeamertemplate{footline}
{
	\begin{beamercolorbox}[ht=3ex,dp=1ex]{erikcolor1}
		\insertshortinstitute[width=.33\textwidth,center] %$\triangleright$
		\insertshortsubtitle[width=.33\textwidth,center] %$\triangleright$
		\insertshortdate[width=.20\textwidth,center] %$\triangleright$
		\hfill\insertframenumber\,/\,\inserttotalframenumber\;\;\;
	\end{beamercolorbox}
}

%Font themes
\usefonttheme{structuresmallcapsserif}
%\usefonttheme[onlysmall]{structurebold}
\usefonttheme{serif}

%Color themes
%\usecolortheme{beetle}
%\usecolortheme{rose}

%Head and foot lines colors
\setbeamercolor{erikcolor1}{fg=white,bg=blue!70!green}
\setbeamercolor{erikcolor2}{fg=white,bg=blue!60!green!10!white}

%Titles color
\setbeamercolor{frametitle}{fg=black,bg=green!30!blue!30!white}
\setbeamercolor{title}{fg=black,bg=green!30!blue!30!white}

%Block color
\setbeamercolor{block title}{fg=white,bg=blue!70!green}
\setbeamercolor{block body}{fg=black,bg=green!30!blue!30!white}

%Background color
\setbeamercolor{background canvas}{bg=}

%Covered items color
\setbeamercovered{transparent}

\AtBeginSection[]
{
   \begin{frame}<beamer>
       \frametitle{Lecture plan}
       \tableofcontents[currentsection,currentsubsection]
   \end{frame}
}



\title{Welcome to logic, reason, and persuasion}
\subtitle{Introductory class meeting}
\author[Hoversten]{Erik Hoversten}
\institute[lrp-f14]{Logic, reason, and persuasion: fall 2014 \\ Rutgers University}
\date[09/03/2014]{September 3, 2014}

\begin{document}

\begin{frame}
\titlepage
\end{frame}

\section{Course outline}

\begin{frame}
\frametitle{Overview}

\begin{block}{In this class...}
...we will introduce a variety of concepts and tools of \textbf{critical thinking}.  The primary tool of philosophical study is the \textbf{argument}.  We'll be examining argumentation from a variety of different perspectives.  We'll begin by introducing the concepts of \textbf{reason} and \textbf{reasoning}.  Once we have the basics down, we'll move on to \textbf{informal analysis} of arguments with an emphasis on understanding and diagnosing \textbf{informal fallacies} of reasoning.  The bulk of the term will be spent on formal argumentation.  We'll explore \textbf{deductive}, \textbf{inductive}, and \textbf{abductive} arguments in depth.
\end{block}

\end{frame}

\begin{frame}
\frametitle{Your instructor}

\begin{block}{Erik Hoversten}
  \begin{tabular}{ll}
    \textbf{Email:} & \href{mailto:ehoversten@philosophy.rutgers.edu}{ehoversten@philosophy.rutgers.edu} \\
    \textbf{Office:} & \href{http://rumaps.rutgers.edu/location/gateway-transit-village}{106 Somerset St}, Room 534, CAC \\
    \textbf{Office Hrs:} & Wed and Thurs 2:00p-3:30p, and by appt. \\
  \end{tabular}
\end{block}

\end{frame}

\begin{frame}
\frametitle{Expectations}

\begin{block}<2->{Assessment}
  \begin{tabular}{lll}
   Assignment & Due date and time & Point value \\ \hline
   Midterm exam & 10/20 in class & 100pts \\
   Final exam & 12/16 8a-11a & 100pts \\
   Homework & Various & 6 @ 20pts = 120pts \\
  \end{tabular}
\end{block}  

\begin{block}<3->{Additional work}
  \begin{itemize}
    \item \textbf{Read} the assigned materials; preferably before the relevant class meeting.
    \item \textbf{Practice} the recommended exercises.
    \item \textbf{Engage} with your classmates using the chat room on the course website and by asking questions (and providing answers) in class.
  \end{itemize}
\end{block}

\end{frame}

\begin{frame}
\frametitle{Course materials}

\begin{block}<2->{Book}
  \href{http://bit.ly/1rw2moM}{A Concise Introduction to Logic}, 12th edition, by Patrick Hurley
\end{block}

\begin{block}<3->{Additional resources}
  Lecture notes and recommended readings available in the \textbf{Reasources} tab on the course website.
\end{block}

\begin{block}<4->{Course website}
\href{https://sakai.rutgers.edu/portal/site/34443d66-a245-4833-953e-1d238502ff28}{LRP 05 F14}: access through Sakai
\end{block}

\end{frame}

\begin{frame}
\frametitle{Proposed schedule}
\tiny

\begin{tabular}{|l|l|l|l|}
\hline
\multirow{3}{*}{Reasons \& reasoning} 
 & W 09/03 & Syllabus & \\
 & M 09/08 & Varieties of reasons & H: \S\S1.1-1.2, LN: reasons \\
 & W 09/10 & Arguments & H: \S\S1.3-1.4 \\

\hline
\multirow{4}{*}{Informal analysis} 
 & M 09/15 & Language \& translations & H: \S\S2.1-2.2 \\
 & W 09/17 & Mood \& content &	\textcolor{red}{HW1 due} \\
 & M 09/22 & Informal fallacies 1 & H: \S\S3.1-3.3 \\
 & W 09/24 & Informal fallacies 2 & \S\S3.4-3.5\\
 & M 09/29 & Buffer session & \\

\hline
\multirow{9}{*}{Deduction} 
 & W 10/01 & Categorical propositions & H: \S\S4.1-4.2, \textcolor{red}{HW2 due} \\
 & M 10/06 & Venn diagrams and syllogisms & H: Ch. 5 \\
 & W 10/08 & Truth functions and connectives & H: \S\S6.1-6.3 \\
 & M 10/13 & Rules of inference 1 & H: Ch. 7 \\
 
 & W 10/15 & Review session & \textcolor{red}{HW3 due} \\
 & M 10/20 & & \textcolor{red}{Midterm exam (in class)} \\
 
 & W 10/22 & Rules of inference 2 & H: Ch. 7 \\
 & M 10/27 & Buffer session & \\
 
\hline 
\multirow{6}{*}{Induction}  
 & W 10/29 & Ampliative reasoning & \textcolor{red}{HW4 due} \\
 & M 11/03 & Analogy & H: Ch. 9 \\
 & W 11/05 & Probability & H: Ch. 11\\
 & M 11/10 & Statistical reasoning & H: Ch. 12 \\
 & W 11/12 & Decision theory & \textcolor{red}{HW5 due} \\
 & M 11/17 & Applications & \\
 & W 11/19 & Buffer session & \\

\hline 
\multirow{3}{*}{Abduction}

 & M 11/24 & Scientific reasoning & H: Ch. 13 \\
 & W 11/26 & & No class \\
 & M 12/01 & The problem of induction & \textcolor{red}{HW6 due} \\
 & W 12/03 & Science and pseudoscience & H: Ch. 14 \\
 & M 12/08 & Applications & \\
 & W 12/10 & Review session & \\
 & T 12/16 & & \textcolor{red}{Final exam (8a-11a)} \\

\hline

\end{tabular}

[H = Hurley, LN = Lecture notes (Sakai), AR = Additional reading (Sakai)]

\end{frame}

\section{What is philosophy?}

\begin{frame}
\frametitle{Common conceptions of philosophy}

\begin{block}<2->{Useless}
  Thinking really hard about stuff, but not actually doing anything.
\end{block}

\begin{block}<3->{Section of the bookstore}
  \begin{itemize}
    \item how to achieve wisdom/understanding/peace/fortune
    \item tied to religion/spirituality/mysticism/faith
  \end{itemize}
\end{block}

\begin{block}<4->{System of belief}
  \begin{itemize}
    \item my philosophy is...
  \end{itemize}
\end{block}

\begin{block}<5->{Academic discipline}
  \begin{itemize}
    \item a field of inquiry
    \item usually aligned with the humanities
  \end{itemize}
\end{block}

\end{frame}
\begin{frame}
\frametitle{Defining a field of inquiry}
\framesubtitle{In general}

\begin{block}<2->{Subject matter}
  \begin{itemize}
    \item geography
    \item baseball analyst
  \end{itemize}
\end{block}

\begin{block}<3->{Purpose}
  \begin{itemize}
    \item repairman
    \item legal adjudicator
  \end{itemize}
\end{block}

\begin{block}<4->{Procedure}
  \begin{itemize}
    \item the game of baseball
  \end{itemize}
\end{block}

\end{frame}

\begin{frame}
\frametitle{Defining a field of inquiry}
\framesubtitle{Philosophy}

\begin{block}<2->{Subject matter}
  \begin{itemize}
    \item epistemology (knowledge)
    \item ethics (right action)
  \end{itemize}
\end{block}

\begin{block}<3->{Purpose}
  \begin{itemize}
    \item discover truth
    \item find peace of mind
  \end{itemize}
\end{block}

\begin{block}<4->{Procedure}
  \begin{itemize}
    \item cooperative exchange of reasons for belief
    \item objective, public standards of argumentation
  \end{itemize}
\end{block}

\end{frame}

\section{Preview of next class}

\begin{frame}
\frametitle{Reason and argument}

\begin{block}<2->{Varieties of reasons}
  \begin{itemize}
    \item epistemic v. pragmatic
    \item explanatory v. justificatory
    \item first blush v. all things considered
  \end{itemize}
\end{block}

\begin{block}<3->{Arguments}
  \begin{itemize}
    \item what is a philosophical argument?
    \item how do we recognize and construct one?
    \item what kinds of argument are there?
  \end{itemize}
\end{block}

\end{frame}

\end{document}
