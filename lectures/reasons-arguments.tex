\documentclass[10pt,letterpaper,xcolor=dvipsnames,handout]{beamer}

%\usepackage[colorlinks=true,linkcolor=blue]{hyperref}
\usepackage{amssymb,mathabx}
\usepackage{linguex}
\usepackage{verbatim,enumerate,multirow}
\usepackage{xcolor}
%\usepackage{floatflt}

\usepackage{pgfpages} % to put several slides on one page
\pgfpagesuselayout{4 on 1}[letterpaper, landscape, border shrink=5mm]

\mode<article>{}

%Template themes
\usetheme{boxes}
%\useoutertheme{miniframes}
\useoutertheme{shadow}

%Templates
\setbeamertemplate{blocks}[rounded][shadow=true]
\setbeamertemplate{navigation symbols}[vertical]
\setbeamertemplate{section in head/foot shaded}[default][20]
\setbeamertemplate{title}

\setbeamertemplate{headline}
{
	\begin{beamercolorbox}[ht=3ex,dp=1ex]{erikcolor1}
		\insertshorttitle
		%\insertsectionnavigationhorizontal{\textwidth}{}{}
		%\usebeamerfont{title in head/foot}
	\end{beamercolorbox}
	\begin{beamercolorbox}[ht=3ex,dp=2ex]{erikcolor2}
	  \insertsectionnavigationhorizontal{\textwidth}{}{}
		%\insertsubsectionnavigationhorizontal{\textwidth}{}{}
		%\insertsubsection
	\end{beamercolorbox}
}
\setbeamertemplate{footline}
{
	\begin{beamercolorbox}[ht=3ex,dp=1ex]{erikcolor1}
		\insertshortinstitute[width=.33\textwidth,center] %$\triangleright$
		\insertshortsubtitle[width=.33\textwidth,center] %$\triangleright$
		\insertshortdate[width=.20\textwidth,center] %$\triangleright$
		\hfill\insertframenumber\,/\,\inserttotalframenumber\;\;\;
	\end{beamercolorbox}
}

%Font themes
\usefonttheme{structuresmallcapsserif}
%\usefonttheme[onlysmall]{structurebold}
\usefonttheme{serif}

%Color themes
%\usecolortheme{beetle}
%\usecolortheme{rose}

%Head and foot lines colors
\setbeamercolor{erikcolor1}{fg=white,bg=blue!70!green}
\setbeamercolor{erikcolor2}{fg=white,bg=blue!60!green!10!white}

%Titles color
\setbeamercolor{frametitle}{fg=black,bg=green!30!blue!30!white}
\setbeamercolor{title}{fg=black,bg=green!30!blue!30!white}

%Block color
\setbeamercolor{block title}{fg=white,bg=blue!70!green}
\setbeamercolor{block body}{fg=black,bg=green!30!blue!30!white}

%Background color
\setbeamercolor{background canvas}{bg=}

%Covered items color
\setbeamercovered{transparent}

\AtBeginSection[]
{
   \begin{frame}<beamer>
       \frametitle{Lecture plan}
       \tableofcontents[currentsection,currentsubsection]
   \end{frame}
}



\title{Reasons \& Argument}
\subtitle{basic categorization}
\author[Hoversten]{Erik Hoversten}
\institute[lrp-f14]{Logic, reason, and persuasion: fall 2014 \\ Rutgers University}
\date[09/08/2014]{September 8, 2014}

\begin{document}

\begin{frame}
\titlepage
\end{frame}

\section{Inquiry}

\begin{frame}
\frametitle{Inquiry}

  \begin{itemize}
    \item<2-> Think of the sum total of an individual's beliefs as determining a \textbf{picture of the world}.
    \item<3-> An \textbf{inquiry} is any process of \textit{forming}, \textit{testing}, and \textit{revising} one's beliefs.
    \item<4-> If all goes well, at the end of inquiry, one has come to hold a more \textit{accurate} picture of the world.
    \item<5-> Inquiry can be an individual process, where one person investigates a part of the world to directly increase her knowledge.
    \item<6-> But commonly, it is done via \textbf{communicative exchange} with other inquirers.
  \end{itemize}
  
\end{frame}

\begin{frame}
\frametitle{Communicative exchange}
\small

\begin{block}{Initial state}<2->
  We all enter any communicative exchange with certain \textbf{presuppositions} (things that we take for granted, our current picture of the world) and \textbf{uncertainties} (questions that we want to get resolved).
\end{block}

\begin{block}{Goal}<3->
  To \textbf{resolve} our uncertainties and \textbf{revise} our presuppositions in a way that provides us with a more accurate picture of the world as a whole.
\end{block}

\begin{block}{Procedure}<4->
  \begin{itemize}
    \item An honest, cooperative exchange of reasons for our beliefs.
    \item This means \textit{offering} what we take to be good reasons for our own beliefs, and honestly \textit{assessing} the reasons that others offer.
    \item When we either accept or reject the beliefs of another inquirer, we want it to be on the basis of the reasons brought to bear in the course of the inquiry.
  \end{itemize}
\end{block}

\end{frame}

\section{Kinds of reasons}

\begin{frame}{What is a reason?}
  The word \textit{reason} can be used in many different ways.  To help get a grip on what it means to engage in the practice of giving and receiving reasons for our beliefs, let's examine some of the different kinds of reasons we might profer for a position.
\end{frame}

\subsection{Subjective v. objective}

\begin{frame}
\frametitle{Subjective v. objective}

Imagine that you are thirsty.  In front of you is a glass that is full of a clear liquid.  The liquid appears to be water. I know (but you don't) that the liquid is gasoline.

\begin{block}<2->{In this scenario, both $A$ and $B$ can be true}
  \begin{enumerate}[$\rightharpoonup$]
    \item $A$: There is reason for you to drink the liquid in the glass.
    \item $B$: There is reason for you to not drink the liquid in the glass.
  \end{enumerate}
\end{block}

\end{frame}

\begin{frame}
\frametitle{Subjective v. objective}

\begin{block}{$A$ references a \textbf{subjective} reason}
  Given the state of your knowledge and your desires, it makes sense for you to drink the liquid.
\end{block}

\begin{block}<2->{$B$ references an \textbf{objective} reason}
  Given the state of the world as it actually is and your desires, it doesn't make sense for you to drink the liquid.
\end{block}

\uncover<3->{Our goal as inquirers is to come to mutually accept \textit{objective} reasons.  But if someone's behavior doesn't make sense from our perspective, it can be valuable to see if we can make sense of it in terms of the \textit{subjective} reasons they have at their disposal.}

\end{frame}

\subsection{Epistemic v. Pragmatic}

\begin{frame}
\frametitle{Epistemic v. Pragmatic}

Imagine that you are being chased through the woods by a bear.  All of a sudden, you break into a clearing and approach the edge of a chasm.  The chasm looks to be about 20ft wide. It occurs to you that if you try to change course, you will most certainly be devoured by the bear.  But you also recall that the best long jumpers in the world rarely jump more than 25ft, and you are no world class athlete. You are in quite a predicament.

\begin{block}<2->{In this scenario, both $C$ and $D$ can be true}
  \begin{enumerate}[$\rightharpoonup$]
    \item $C$: You have reason to believe that you cannot make the jump.
    \item $D$: You have reason to believe that you can make the jump.
  \end{enumerate}
  
\end{block}

\end{frame}

\begin{frame}
\frametitle{Epistemic v. Pragmatic}

\begin{block}{$C$ references an \textbf{epistemic} reason}
  \begin{itemize}
    \item All of your evidence suggests that the distance is too far for you to cross in a single leap. 
    \item Epistemic reasons appeal to our \textbf{cognitive faculties} (beliefs) and purport to support the \textit{truth} of some claim.
  \end{itemize}
\end{block}

\begin{block}<2->{$D$ references a \textbf{pragmatic} reason}
  \begin{itemize}
    \item If you don't jump, you're gonna get eaten.  
    \item Maybe believing you can make it will give you a slightly better chance of doing so. 
    \item So, you might as well believe it. 
    \item Pragmatic reasons appeal to our \textbf{connative faculties} (our senses of emotion and well-being) and purport to show that we \textit{benefit} in some way from believing the claim.
  \end{itemize}
\end{block}

\end{frame}

\subsection{Explanatory v. Justificatory}

\begin{frame}
\frametitle{Explanatory v. Justificatory}

One role that reasons play is that they serve as answers to \textit{Why?} questions.  But \textit{Why?} questions can be ambiguous.  Consider:

\ex. Why did Frank take the marble rye home?
  \a. Because he paid for it.
  \b. Because he thought no one was looking.


\end{frame}


\begin{frame}
\frametitle{Explanatory v. Justificatory}

\begin{block}{Imagine that \Last[a] is true.} 
This response provides a reason for why Frank took the beer, and supposing that having paid for something gives someone the right to do with it what they please, it is a reason that \textbf{justifies} Frank's behavior.
\end{block}

\begin{block}<2->{But now suppose that \Last[b] is true (and \Last[a] is false).}
Then, this response also provides a reason for Frank's behavior, but it doesn't seem to justify that behavior.  Instead, one might offer this reason as a way of \textbf{explaining} why Frank did it. We might follow it up by saying, ``But he should have left it there.''
\end{block}

\end{frame}

\subsection{First blush v. All things considered}

\begin{frame}
\frametitle{First blush v. All things considered}

Holmes is investigating a crime scene. He first comes upon a blond hair. He knows that of his two suspects, only Sally has blond hair, whereas Dirk is a brunette.  This leads Holmes to say:

\uncover<2->{\ex. I now have reason to believe that Sally is the culprit.\label{hair}}

\uncover<3->{As he investigates further, he finds a solid fingerprint.  His friend at the police station tells him that there is no strong correlation with the fingerprint of Sally's that they have on file.  Since fingerprint evidence is more weighty than hair folicles,
%\footnote{How much more weighty? It's not clear.  \href{http://en.wikipedia.org/wiki/Finger_Printing\#Validity}{Finger printing} is less of a science than many people realize.} 
Holmes now says:}

\uncover<4->{\ex. Now the balance of reasons suggests that Sally did not do it.\label{nsally}}

\uncover<5->{More searching uncovers a bottle of brown hair dye in the trash and a man's footprint at the scene. Having completed a thorough search, Holme's exclaims:}

\uncover<6->{\ex. Aha! It wasn't Sally at all, but Dirk!\label{dirk}}

\end{frame}

\begin{frame}
\frametitle{First blush v. All things considered}

\begin{block}{\ref{hair} seems accurate at first blush.}
  \begin{itemize}
    \item But two things happen as Holmes looks into the case in more detail.
    \item<2-> First he gets \textbf{rebutting evidence} against \ref{hair}.  The fingerprint evidence stands in direct conflict with what the hair seems to show. This leads him to say \ref{nsally}.
    \item<3-> Then he gets \textbf{under cutting} evidence against \ref{hair}. When he discovers the hair dye, he realizes that a blond hair doesn't support Sally's guilt at all, since Dirk very well could have been blond, too. This leads him to say \ref{dirk}.
  \end{itemize} 
\end{block}

\begin{block}<4->{Putting all the pieces together...}
  Holmes forms an \textbf{all things considered} opinion that Dirk is the criminal. The sum total of the evidence he has leads him to revise his initial assessment of the case.
\end{block} 

\end{frame}

\begin{frame}
\frametitle{What kind of things are reasons?}

\begin{block}{Objectual reasons}
  \begin{itemize}
    \item Reasons are related to evidence.
    \item And often we count various \textbf{objects} as evidence.
    \item e.g.: The smoking gun, the blood stain on the carpet, the fingerprint
  \end{itemize}
\end{block}

\begin{block}<2->{Propositional reasons}
  \begin{itemize}
    \item But reasons are also often given using ``because'' statements.
    \item I was late to class today \textit{because} my roommate shut off my alarm clock.
    \item In this case, I offer a \textbf{proposition} as my reason.
    \item Propositions are the content (or meaning) of \textbf{statements}. They make some claim about the way the world is.
    \item They often have a clearly defined \textbf{subject} and \textbf{predicate}.
  \end{itemize}
\end{block}

\end{frame}

\section{Arguments}

\begin{frame}
\frametitle{Arguments as reasons}

\begin{enumerate}
  \item In this class, we're going to try to put all our \textbf{reasons} that we give for various beliefs into the form of an \textbf{argument}.
  \item<2-> The purpose of this practice is to make it clear what exactly we are \textbf{offering} as our reason, and also to make it easier to \textbf{assess} whether our reason is a good one.
  \item<3->\href{https://www.youtube.com/watch?v=kQFKtI6gn9Y}{So, what is an argument?}
\end{enumerate}
  
\end{frame}

\subsection{Basic structure of arguments}

\begin{frame}
\frametitle{Argument form}

\begin{block}{A philosophical argument is...}
  \begin{itemize}
    \item<2-> A set of propositions
      \begin{itemize}
        \item One of these is labeled the \textbf{conclusion}.
        \item The rest are called the \textbf{premises}.
      \end{itemize}
    \item<3-> A relation among the propositions
      \begin{itemize}
        \item The premises are said to \textbf{support} the conclusion.
        \item We can also say that the conclusion is believed as an \textbf{inference} from the premises.
      \end{itemize}
  \end{itemize}
\end{block}
  
\uncover<4->{In order to provide a complete reason in favor of your conclusion, you need to provide both the supporting premises \textbf{and} the nature of the support those premises provide for the conclusion.  When you've done that, you've given your \textbf{reasoning} in favor of the conclusion.} 
 
\end{frame}

\begin{frame}
\frametitle{Example argument}

\begin{block}{Example}
  \begin{itemize}
    \item Premise 1: John can only be in one of three places: his office, the common room, or his classroom.
    \item Premise 2: He's not in his office.
    \item Premise 3: He's not in the common room.
    \item Conclusion: So, John is in his classroom.
  \end{itemize}
\end{block}

\uncover<2->{The inference in this argument is a \textbf{process of elimination}.}

\end{frame}

\subsection{Types of arguments}

\begin{frame}
\frametitle{Types of argument}

We can distinguish different kinds of argument based on the nature of the support the premises are said to provide for the conclusion.

\begin{block}<2->{Deductive}
  In a deductive argument, the premises provide \textbf{conclusive} support for the conclusion.  If the premises are actually true, then \textbf{there's no way} the conclusion can be false.
\end{block}

\begin{block}<3->{Inductive}
  In an inductive argument, the premises make it \textbf{more likely} that the conclusion is true.  But they don't guarantee that the conclusion is true.
\end{block}

\begin{block}<4->{Abductive}
  In an abductive argument, the premises provide a \textbf{good explanation} of why the conclusion is true. But even a good explanation can sometimes be incorrect.
\end{block}

\end{frame}

\begin{frame}
\frametitle{Examples}
\small

\begin{block}<2->{Deductive}
  \begin{enumerate}
    \item John is taller than Bill.
    \item Bill is taller than George.
    \item So, John is taller than George.
  \end{enumerate}
\end{block}

\begin{block}<3->{Inductive}
  \begin{enumerate}
    \item Most people who enjoy tofu are vegetarians.
    \item Frank enjoys tofu.
    \item So, Frank is probably a vegetarian.
  \end{enumerate}
\end{block}

\begin{block}<4->{Abductive}
  \begin{enumerate}
    \item Every time there is a lunar eclipse, the Earth's shadow appears round.
    \item If the Earth is a sphere, it would make sense that it's shadow is always round.
    \item So, the Earth must be a sphere.
  \end{enumerate}
\end{block}

\end{frame}

\begin{frame}
\frametitle{Uses for the argument types}

\begin{block}{Deductive}
Deductive arguments are useful in mathematics, and perhaps in legal settings where the standards of argumentation are incredibly high.
\end{block}

\begin{block}{Inductive}
Inductive arguments are useful in the sciences, and in applications like weather forcasting, where we want to know the chances of an outcome even if we can't be certain that it will happen.
\end{block}

\begin{block}{Abductive}
Abductive arguments are useful in the sciences, and in group brainstorming sessions, where we may not even know what all the alternatives are, and we want to get the options out on the table.
\end{block}

\end{frame}

\begin{frame}
\frametitle{The good, the bad, and the ugly}
\small

\begin{block}{Today, we looked at a \textbf{taxonomy} (categorization) of reasons and arguments.}
\begin{itemize} 
  \item But we shouldn't think of one kind of argument as \textit{better} than another.
  \item Different argumentation techniques are appropriate for different situations.
  \item What is important is that we try to be clear on what sort of reason we are faced with, so that we can better understand how to revise our beliefs in response to it.
\end{itemize}
\end{block}

\begin{block}<2->{But not all arguments are created equal}
\begin{itemize}
  \item Over the course of the semester, we will examine ways of assessing arguments to determine whether we should be swayed by them.
  \item But we can only do this if we take the time to lay out the argument faithfully and understand what kind of reason our fellow inquirer has presented us with.
\end{itemize}
\end{block}

\end{frame}

\begin{frame}
\frametitle{Next meeting}

Next meeting we're going to look more closely at argument structure. We'll examine:
\begin{itemize}
  \item Types of propositions
  \item Properties of arguments
  \item Basic assessment criteria for arguments
\end{itemize}

\begin{block}<2->{Relevant reading}
  Hurley: \S\S 1.3-1.4
\end{block}

\end{frame}

\end{document}
