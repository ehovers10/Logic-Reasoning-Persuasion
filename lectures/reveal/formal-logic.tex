<!doctype html>
<html lang="en">

	<head>
		<meta charset="utf-8">

		<title>reveal.js - The HTML Presentation Framework</title>

		<meta name="description" content="A framework for easily creating beautiful presentations using HTML">
		<meta name="author" content="e-rizzle" >

		<meta name="apple-mobile-web-app-capable" content="yes" />
		<meta name="apple-mobile-web-app-status-bar-style" content="black-translucent" />

		<meta name="viewport" content="width=device-width, initial-scale=1.0, maximum-scale=1.0, user-scalable=no">

		<link rel="stylesheet" href="css/reveal.css">
		<link rel="stylesheet" href="css/theme/default.css" id="theme">

		<!-- For syntax highlighting -->
		<link rel="stylesheet" href="lib/css/zenburn.css">

		
		
		<!-- If the query includes 'print-pdf', include the PDF print sheet -->
		<script>
			if( window.location.search.match( /print-pdf/gi ) ) {
				var link = document.createElement( 'link' );
				link.rel = 'stylesheet';
				link.type = 'text/css';
				link.href = 'css/print/pdf.css';
				document.getElementsByTagName( 'head' )[0].appendChild( link );
			}
		</script>

		<!--[if lt IE 9]>
		<script src="lib/js/html5shiv.js"></script>
		<![endif]-->
	</head>

	<body>

		<div class="reveal">

			<!-- Any section element inside of this container is displayed as a slide -->
			<div class="slides">

<section id="what-is-formal-logic" class="titleslide slide level1">
	<h1>What is formal logic?</h1>
</section>

<section id="arguments-on-different-subjects" class="slide level3">
<h1>Arguments on different subjects</h1>
<p><span>Physics</span></p>
<ul>
<li><p>Not both Supersymmetry and the Multiverse hypothesis can be correct.</p></li>
<li><p>So, we should develop tests to determine which one is wrong.</p></li>
</ul>
<p>&lt;2-&gt;<span>Politics</span></p>
<ul>
<li><p>Neither Williams nor Hanson are viable candidates in this race.</p></li>
<li><p>So, the party will lose the seat with Williams, but they’ll lose it with Hanson, too.</p></li>
</ul>
<p>&lt;3-&gt;<span>Gossip</span></p>
<ul>
<li><p>Laura and Jerry don’t both show up anywhere.</p></li>
<li><p>I saw Laura at the party last night, so you know Jerry was nowhere to be seen.</p></li>
</ul>
</section>

<section id="topic-neutrality" class="slide level3">
<h1>Topic neutrality</h1>
<p><span>The arguments on the previous slide cover a variety of topics.</span></p>
<ul>
<li><p>but they exhibit the same <strong>logical structure</strong></p></li>
<li><p>we say that logic is <strong>topic neutral</strong> because it applies no matter what subject you are talking about.</p></li>
<li><p>Humans can reason about a wide variety of things, but our minds rely on a small set of <strong>tools</strong> that we apply to all these different things.</p></li>
</ul>
<p>&lt;2-&gt;<span>Formal logic</span></p>
<ul>
<li><p>Formal logic investigates these tools of reasoning.</p></li>
<li><p>It looks only at the <strong>logical structure</strong> of propositions and arguments.</p></li>
<li><p>It ignores the specific content of the propositions involved in the argument.</p></li>
</ul>
</section>

<section id="abstraction" class="slide level3">
<h1>Abstraction</h1>
<p><span>Form</span></p>
<ul>
<li><p>Topic neutrality means that the specific content of the propositions involved doesn’t matter for the logic that is in play.</p></li>
<li><p>What matters is the structure (or form) of the argument.</p></li>
<li><p>Formal logic is the study of how different forms relate to each other. We will define <em>validity</em> in these terms.</p></li>
</ul>
<p>&lt;2-&gt;<span>Replacement</span></p>
<ul>
<li><p>We <em>abstract</em> away from specifics by replacing the propositional <em>content</em> with propositional <strong>variables</strong> (<span class="math"><em>A</em>, <em>B</em>, <em>C</em>, …</span>)</p></li>
<li><p>We just have to make sure that we use the same variable every time a particular proposition appears in the argument.</p></li>
<li><p>Then, we can treat as the same any argument that looks the same after this <strong>replacement procedure</strong>.</p></li>
</ul>
</section><section id="the-structure-of-arguments" class="slide level3">
<h1>The structure of arguments</h1>
<p><span>Physics</span></p>
<ul>
<li><p>&lt;2-&gt;<span>Not</span> &lt;4-&gt;<span>both</span> Supersymmetry &lt;4-&gt;<span>and</span> the Multiverse hypothesis can be correct.</p></li>
<li><p>So, we should develop tests to determine &lt;3-&gt;<span>which one</span> is &lt;2-&gt;<span>wrong</span>.</p></li>
</ul>
<p><span>Politics</span></p>
<ul>
<li><p>&lt;2-&gt;<span>Neither</span> Williams &lt;2-&gt;<span>nor</span> Hanson are viable candidates in this race.</p></li>
<li><p>So, the party will lose the seat with Williams, &lt;4-&gt;<span>but</span> they’ll lose it with Hanson, &lt;4-&gt;<span>too</span>.</p></li>
</ul>
<p><span>Gossip</span></p>
<ul>
<li><p>Laura &lt;4-&gt;<span>and</span> Jerry &lt;2-&gt;<span>don’t</span> &lt;4-&gt;<span>both</span> show up anywhere.</p></li>
<li><p>I saw Laura at the party last night, &lt;4-&gt;<span>so</span> you know Jerry was &lt;2-&gt;<span>nowhere</span> to be seen.</p></li>
</ul>
</section><section id="demorgans-rules" class="slide level3">
<h1>Demorgan’s rules</h1>
<p>With some finagling, we can perform a replacement procedure on each of the arguments above to get one of the following argument structures.</p>
<p><span>DeMorgan’s rule 1</span></p>
<ul>
<li><p>not-( <span class="math"><em>A</em></span> and <span class="math"><em>B</em></span> )</p></li>
<li><p><span class="math">∴ </span>, ( not-<span class="math"><em>A</em></span> or not-<span class="math"><em>B</em></span> )</p></li>
</ul>
<p>&lt;2-&gt;<span>DeMorgan’s rule 2</span></p>
<ul>
<li><p>not-( <span class="math"><em>A</em></span> or <span class="math"><em>B</em></span> )</p></li>
<li><p><span class="math">∴ </span>, ( not-<span class="math"><em>A</em></span> and not-<span class="math"><em>B</em></span> )</p></li>
</ul>
</section><section id="the-physics-example" class="slide level3">
<h1>The physics example</h1>
<p><span>Physics</span></p>
<ul>
<li><p>Supersymmetry the Multiverse hypothesis can be correct.</p></li>
<li><p>So, we should develop tests to determine is .</p></li>
</ul>
<p>&lt;2-&gt;<span>Adjusted</span></p>
<ul>
<li><p>(Supersymmetry is correct the Multiverse hypothesis is correct).</p></li>
<li><p>So, we should develop tests to determine: Supersymmetry is correct the Multiverse hypothesis is correct.</p></li>
</ul>
</section><section id="replacement-procedure" class="slide level3">
<h1>Replacement procedure</h1>
<p><span>Variable assignment</span></p>
<ul>
<li><p>Let <span class="math"><em>A</em></span> stand for “Supersymmetry is correct”</p></li>
<li><p>Let <span class="math"><em>B</em></span> stand for “the Multiverse hypothesis is correct”</p></li>
</ul>
<p>&lt;2-&gt;<span>Replaced</span></p>
<ul>
<li><p>Not (<span class="math"><em>A</em></span> and <span class="math"><em>B</em></span>).</p></li>
<li><p>So, we should develop tests to determine: Not-<span class="math"><em>A</em></span> or not-<span class="math"><em>B</em></span>.</p></li>
</ul>
</section><section id="logical-systems" class="slide level3">
<h1>Logical systems</h1>
<p>In addition to abstracting over whole propositions, we can also abstract over parts of propositions. What things we abstract over depends on the vocabulary of the particular logical system we are using.</p>
<p><span>Logical vocabulary</span></p>
<ul>
<li><p>These are the topic neurtal bits that do the work in a logical system. We don’t abstract over these.</p></li>
<li><p>Propositional logic: <span class="math"><em>n</em><em>o</em><em>t</em></span>, <span class="math"><em>a</em><em>n</em><em>d</em></span>, <span class="math"><em>o</em><em>r</em></span>, <span class="math"><em>i</em><em>f</em>…<em>t</em><em>h</em><em>e</em><em>n</em></span></p></li>
</ul>
<p>&lt;2-&gt;<span>Non-logical vocabulary</span></p>
<ul>
<li><p>There are the topic specific bits. They don’t do any work in the logic, so we abstract over them.</p></li>
<li><p>Propositional logic: “John went to the park”, “Time is eternal”, “Curling is boring”</p></li>
</ul>
</section></section>
<section><section id="categorical-logic" class="titleslide slide level1"><h1>Categorical logic</h1></section><section id="categorical-logic-1" class="slide level3">
<h1>Categorical logic</h1>
<p><span>What’s it about?</span></p>
<ul>
<li><p>The first logical system that we are going to investigate in depth is called <em>categorical logic</em></p></li>
<li><p>It pertains to reasoning about <strong>relations between classes</strong> (groups) of individuals.</p></li>
<li><p>It was devised by Aristotle (and others) over 2000 years ago.</p></li>
<li><p>Logicians in the middle ages spent a lot of time categorizing and developing the rules of the logical system.</p></li>
<li><p>Comparison of classes is a basic tool used in reasoning, so the logic dedicated to it is still of great interest today.</p></li>
</ul>
</section><section id="the-vocabulary" class="slide level3">
<h1>The vocabulary</h1>
<p><span>Logical vocabulary</span></p>
<ul>
<li><p>Quantifiers: <span class="math"><em>a</em><em>l</em><em>l</em></span>, <span class="math"><em>s</em><em>o</em><em>m</em><em>e</em></span>, <span class="math"><em>n</em><em>o</em></span></p></li>
<li><p>Copulas: <span class="math"><em>i</em><em>s</em></span>, <span class="math"><em>a</em><em>r</em><em>e</em></span>, <span class="math"><em>i</em><em>s</em><em>n</em><em>o</em><em>t</em></span>, <span class="math"><em>a</em><em>r</em><em>e</em><em>n</em><em>o</em><em>t</em></span></p></li>
<li><p>Every categorical proposition contains one quantifier and one copula.</p></li>
</ul>
<p>&lt;2-&gt;<span>Non-logical vocabulary</span></p>
<ul>
<li><p>Subject terms: “Americans”, “Happy people”, “Worms”</p></li>
<li><p>Predicate terms: “greedy”, “red”, “go to Rutgers”</p></li>
<li><p>These terms all pick out a class (or group) of individuals.</p></li>
<li><p>Every categorical proposition contains one subject term and one predicate term.</p></li>
<li><p>The same word can often serve as either a subject term or a predicate term.</p></li>
</ul>
</section><section id="proposition-schema" class="slide level3">
<h1>Proposition schema</h1>
<p>Given the logical vocabulary of categorical logic, there are only 4 types of categorical proposition. We call these types proposition <strong>schema</strong>.</p>
<dl>
<dt>All S are P</dt>
<dd><p>the whole subject class is included in the predicate class.</p>
</dd>
<dt>No S are P</dt>
<dd><p>the whole subject class is excluded from the predicate class.</p>
</dd>
<dt>Some S are P</dt>
<dd><p>a part of the subject class is included in the predicate class.</p>
</dd>
<dt>Some S are not P</dt>
<dd><p>a part of the subject class is excluded from the predicate class.</p>
</dd>
</dl>
</section><section id="types-of-categorical-proposition" class="slide level3">
<h1>Types of categorical proposition</h1>
<p>Categorical propositions can be grouped on the basis of their <strong>quality</strong> and their <strong>quantity</strong></p>
<p><span>Quality</span></p>
<ul>
<li><p>This marks whether the proposition affirms or denies class membership.</p></li>
<li><p>Affirmative: “All S are P”, “Some S are P”</p></li>
<li><p>Negative: “No S are P”, “Some S are not P”</p></li>
</ul>
<p><span>Quantity</span></p>
<ul>
<li><p>This marks whether the proposition is universal or existential.</p></li>
<li><p>Universal: “All S are P”, “No S are P”</p></li>
<li><p>Existential: “Some S are P”, “Some S are not P”</p></li>
</ul>
</section></section>
<section><section id="next-meeting" class="titleslide slide level1"><h1>Next meeting</h1></section><section id="whats-coming-up" class="slide level3">
<h1>What’s coming up?</h1>
<ul>
<li><p>We’ll continue investigating the properties of categorical logic.</p></li>
<li><p>We’ll introduce some tools for evaluating inferences involving categorical propositions.</p></li>
<li><p>Relevant reading: §§4.3-4.5</p></li>
<li><p>HW #2 is due at the beginning of class</p></li>
<li><p>HW #1 will <strong>definitely</strong> be returned!</p></li>
</ul>
</section></section>

			</div>

		</div>

		<script src="lib/js/head.min.js"></script>
		<script src="js/reveal.min.js"></script>

		<script>

			// Full list of configuration options available here:
			// https://github.com/hakimel/reveal.js#configuration
			Reveal.initialize({
				controls: true,
				progress: true,
				history: true,
				center: true,

				theme: Reveal.getQueryHash().theme, // available themes are in /css/theme
				transition: Reveal.getQueryHash().transition || 'default', // default/cube/page/concave/zoom/linear/fade/none

				// Parallax scrolling
				// parallaxBackgroundImage: 'https://s3.amazonaws.com/hakim-static/reveal-js/reveal-parallax-1.jpg',
				// parallaxBackgroundSize: '2100px 900px',

				// Optional libraries used to extend on reveal.js
				dependencies: [
					{ src: 'lib/js/classList.js', condition: function() { return !document.body.classList; } },
					{ src: 'plugin/markdown/marked.js', condition: function() { return !!document.querySelector( '[data-markdown]' ); } },
					{ src: 'plugin/markdown/markdown.js', condition: function() { return !!document.querySelector( '[data-markdown]' ); } },
					{ src: 'plugin/highlight/highlight.js', async: true, callback: function() { hljs.initHighlightingOnLoad(); } },
					{ src: 'plugin/zoom-js/zoom.js', async: true, condition: function() { return !!document.body.classList; } },
					{ src: 'plugin/notes/notes.js', async: true, condition: function() { return !!document.body.classList; } }
				]
			});

		</script>

	</body>
</html>
