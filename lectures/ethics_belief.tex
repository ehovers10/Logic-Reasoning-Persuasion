\documentclass[letterpaper,10pt]{article}

\usepackage[colorlinks=true,linkcolor=blue]{hyperref}
\usepackage{amssymb}
\usepackage{linguex}

%Margin settings
\usepackage{geometry}
\geometry{hmargin={.75in,.75in},vmargin={1in,1in}}

%Header settings
\usepackage{fancyhdr}
%\setlength{\headheight}{15pt}
\pagestyle{fancy}
\fancyhead{}
\fancyhead[L]{Phil 103, s12}
\fancyhead[C]{Belief and objectivity}
\fancyhead[R]{Hoversten}

%Paragraph settings
\setlength{\parindent}{0pt}
\setlength{\parskip}{2ex plus .5ex minus .2ex}

\begin{document}

\section{Introduction}

Many people seem to think that the question of whether God exists cannot be answered via investigation of objective arguments.  Some people who subscribe to this claim seem to think that the reason objective arguments will fail is that there is no objectively right answer to the question.  In this lecture, we'll look into what it could mean for the following claim to be true:

\begin{quote}
 There is no objectively right answer to the question of whether or not to believe that God exists.
\end{quote}

In particular, we will look at three ways of understanding this claim:
\begin{enumerate}
 \item The sentence ``God exists'' has no objective truth value because ``God'' is a \textbf{context sensitive} expression.
 \item The sentence ``God exists'' has no objective truth value because its truth is \textbf{relative}.
 \item Whether one should or shouldn't believe in God has no objective answer because individuals can believe whatever they want to.
\end{enumerate}

\subsection{Context sensitivity}
In the various arguments for and against the existence of God, we come across a variety of conceptions of what God is like.  In the cosmological argument, God was taken to be the \textit{first cause} in the universe. In the teleological argument, God was taken to be the \textit{intelligent designer} of the universe. And in the problem of evil, God was taken to be an \textit{omnipotent, omnibenevolent creator} of the universe.

The fact that all these different conceptions exist might lead someone to think that the various arguments we've looked at are just talking about different things.  And one might go so far as to claim that there can be no objective answer to whether God exists because ``God'' has no objective meaning.

We can understand this idea better by introducing a tool from the philosophy of language that is valuable in understanding the way communication of ideas works in normal conversations.  We say that a term is \textbf{context sensitive} when the value that it contributes to the meaning of a sentence depends on the \textit{context} in which the term is used. 

As an example, consider the following simple dialog:
\ex. \a.[Gillian: ] ``I'm hungry.''
\b.[David: ] ``I'm not hungry.''

On their face, the sentences uttered by Gillian and David are in direct contradiction; their only difference is that one has a negation that the other lacks. However, we understand these sentences to be perfectly compatible.  The reason is that the two instances of ``I'' pick out different individuals. In Gillian's sentence, \textit{I} picks out Gillian, and in David's sentence, \textit{I} picks out David. This works because the meaning of \textit{I} is context sensitive.  Its contribution to the sentence depends on \textbf{who is speaking}, which is one feature of the context in which the sentence is uttered.

In effect, what Gillian and David are really saying is:
\ex. \a.[Gillian: ] ``Gillian is hungry.''
\b.[David: ] ``David is not hungry.''

And these two sentences are perfectly compatible.

\subsubsection{Application to ``God exists.''}
One might think that the term \textit{God} is context sensitive in the same way that \textit{I} is.  That would mean that the following dialog could be perfectly compatible.

\ex. \a.[Paley: ] ``God exists.''
\b.[Mackie: ] ``God doesn't exist.''

Since Paley and Mackie invoke different ideas with their use of \textit{God}, their sentences are not negations of each other.  But notice that if \textit{God} is context sensitive, then we should be able to substitute in its meaning in each context, like we did with \textit{I}.

\ex. \a.[Paley: ] ``An intelligent designer exists.''
\b.[Mackie: ] ``An omnipotent, omnibenevolent creator doesn't exist.''

These two sentences aren't in direct conflict, but each seems to be making an explicit claim about the world.  Thus, we should be able to debate whether something fitting each concept of God genuinely exists.  So, even if the term \textit{God} is context sensitive, we can still engage in objective argumentation.  We just need to be careful to spell out exactly what we mean when we use the term \textit{God}.

\subsection{Relative truth}
Another way that someone might try to challenge the suggestion that the question whether God exists has an objective answer is to claim that while the sentence ``God exists'' has a single meaning, whether it is true is relative to who is assessing the sentence.  This sort of idea has some precedent in other areas. Take, for instance, the following sentence:

\ex. ``Tofu is yummy.''

When someone utters a sentence like \Last, they usually intend to make an \textbf{evaluative} claim about the world.  Evaluative claims differ from \textbf{factual} claims in that they seem to say as much about the person making the claim as about the world itself.  Tofu has certain objective properties like \textit{being spongy} and \textit{being offwhite}, but the property \textit{being yummy} doesn't seem to be something specific about the tofu itself.  Because of this, we are often willing to say that sentences such as \Last might be \textbf{true for you} but \textbf{false for me}.

\subsubsection{Application to ``God exists.''}
So, might ``God exists'' be a sentence that is true for one person but not true for another?  When people get frustrated over arguing with another about the existence of God, they will sometimes throw their hands up in the air and say that sort of thing.  But there are a couple reasons to think that this isn't the right way to understand the claim that God exists.
\begin{itemize}
 \item First, \textit{exists} doesn't seem to be an evaluative property in the way that \textit{is yummy} is. To say that something exists seems equivalent to making a factual claim about what there is in the world. It's difficult to see how something could exist for one person but not for another.\footnote{It's true that we will often say that something exists for some person that doesn't exist for us.  For instance, we might be willing to say that, for some four year old, Santa Claus exists.  But when we say this we mean that the four year old \textit{thinks} Santa exists.  We don't mean that Santa actually exists in any form.}
 
 \item Second, the way in which people put forward claims about God's existence doesn't seem to match what they do with claims about yumminess.  People might argue about which things are tasty or not, but if others protest, they very quickly decide to ``agree to disagree''. 

 But the matter is different when the question of God's existence comes up.  Devout people are generally far less willing to simply concede to the atheist that God doesn't exist \textit{for them}.  And atheists are rarely willing to grant the believer that God genuinely exists \textit{for them}. 
\end{itemize}

These considerations don't settle the issue, but they do put pressure on the person who claims that ``God exists'' is relative to explain what exactly they mean by that claim.

\subsection{Just an opinion}
Some people seem to think that arguing about the existence of God is futile because whether God exists is just a matter of one person's opinion. In saying this, they seem to suggest that one's belief in God (or their non-belief) isn't subject to rational criticism. Obviously, if it doesn't matter what you believe, then there's no point in arguing over it.  But does it matter what you believe?

In \textit{The Ethics of Belief}, W. K. Clifford argues that it does matter.  In what follows, we'll examine his argument in detail.

\subsubsection{Thought experiments}
%Clifford's argument is an example of a common philosophical practice.  He starts by introducing a \textbf{hypothetical scenario}, and then he invites us to draw certain conclusions on the basis of this example case. The idea is that by thinking about these imagined cases, we will come to understand something about our topic (in this case the notion of \textit{belief}) that we might not have noticed.

\paragraph{Case 1} Clifford asks us to imagine an owner of a ship. The ship owner is approached by a group of would be emigrants who would like to use his ship to move to a new land. The ship is very old, and the owner doesn't know whether it is sea-worthy.  But rather than get it inspected, he quells his worries, puts his faith in Providence, and comes to believe that the ship will make the voyage. In the middle of the journey, the ship sinks and all on board drown.

\paragraph{Case 2} Clifford asks us to imagine an isolated island community in which everyone holds non-standard religious beliefs. A group of outsiders hears a rumor that the priests on the island have been abducting children and brainwashing them to believe their weird religious views.  This, supposedly, is why everyone on the island believes what they do.  The outsiders believe this rumor and start a campaign to turn the public against the priests and unseat them from their position of power.  Upon an inquiry, it is found that the rumors were unfounded; the priests were not doing anything illicit regarding the youth of the island.

For each of these cases, Clifford thinks we all will agree that the main character has done something wrong. In case 1, the ship owner is guilty for the deaths of the passengers.  In case 2, the outsiders are guilty for the character defamation of the priests of the island.

Clifford thinks that what is wrong in these cases is that the main character has formed a belief without adequate evidence for it. He suggests that we have the assessments we do because one only has a \textbf{right to believe} when they have sufficient evidence. From these thought experiments, he draws the following principle:

\begin{quote}
 It is wrong, always, everywhere, and for anyone, to believe anything upon insufficient evidence.
\end{quote}

In the remainder of his paper, Clifford attempts to further support this claim by answering potential objections to it.

\subsubsection{Objections and replies}

\paragraph{Objection 1} In each of the cases Clifford looks at, the belief the main character forms is false. Maybe what is wrong is that they believe falsely, not that they believe without evidence.

\paragraph{Response 1} Here Clifford modifies his thought experiments. He imagines that the ship owner goes through the same process but that the ship (miraculously) survives the voyage.  Clifford claims that even when we change the case in this way, we still think that the ship owner has done something wrong.  If so, then the truth of the belief is not relevant.

\paragraph{Objection 2} In each of the cases, the main character performs an action that results in harm to other people.  Maybe what is wrong is that they perform a bad action, not that they believe without evidence.

\paragraph{Response 2} Clifford has a couple responses to this objection:
\begin{enumerate}
 \item It is impossible to separate the action from the belief. Thus, if one condemns the action, she also condemns the belief.

 Clifford's idea here is that actions and beliefs are closely tied. In general, one's beliefs tend to lead to actions. Thus, if an action is deemed wrong, it is usually based on a belief, which can also be deemed wrong.

 \item We have no trouble criticizing people for their actions, but it might seem a bit weird to criticize someone merely for believing something. To alleviate this worry, Clifford tries to give a reason to think that beliefs are just as morally significant as actions.

 Beliefs have a social basis.  This is seen in a couple of ways.  First, what we believe tends to impact how we act.  And since our actions impact other people, our beliefs thereby impact other people.

 Second, the very expression of our beliefs relies on the linguistic community of which we are a part. Thus, the way we report our beliefs depends on our social environment.  Even further, the ways in which we think are influenced by the social groups in which our thoughts develop. Since our beliefs play a primary role in our thought, our beliefs are influenced by our social environment.

 The upshot here is that belief is as much a social activity as action. This means that beliefs can be subject to moral criticism in the same way that actions are.
\end{enumerate}

\paragraph{Objection 3} Clifford's principle is much too strong.  After all, we can never \textit{know} that our beliefs are true.  If we required certainty before we could believe, we would never believe anything, and we would never act.

\paragraph{Response 3} This objection misconstrues the strength of the principle. The principle suggests that to have a right to one's belief, that person must have \textit{sufficient evidence}, but sufficient evidence doesn't mean \textit{certainty}. Consider the following revised case.

\begin{quote}
 Before the ship owner sends the emigrants on their adventure, he gets the boat checked out by an inspector. The inspector gives it a thorough review and determines that it is sea-worthy.  With this assurance in hand, the ship owner sends it off to sea. In the middle of the journey, the ship sinks and all on board drown.
\end{quote}

Clifford would agree that in this case, we shouldn't hold the ship owner responsible.  But the reason for this is that the ship owner did what he could to aquire the evidence that his boat was safe. It's true that the ship owner doesn't \textit{know} that the ship is safe. But because he has done his due dilligence to justify his belief that it is safe, he is not responsible for its sinking.

\paragraph{Objection 4} In both of the cases presented, the beliefs in question are extremely \textit{momentous}.  Since there is so much at stake, it is important to seek adequate evidence.  But what about ordinary everyday beliefs? Why should we worry about seeking evidence for those?

\paragraph{Response 4} Clifford suggests that believing is habit forming. When we believe things without adequate evidence, we develop a \textbf{credulous character}, which makes us more willing to believe things in the future. Thus, we should avoid willy-nilly belief, even on minor details, in order to avoid becoming too credulous.

Being overly credulous might be detrimental to one's own wellbeing, but Clifford maintains that it is an even more serious failing that that. As a society, we rely on the testimony of others.  It isn't feasible for us to research every question we come across on our own, so we look to others to help us sort out what to believe on various matters. But the success of this practice requires us to be able to \textbf{trust} those to whom we look. We expect that they have adequate evidence for what they are telling us.  Those who form beliefs in the absence of adequate evidence undermine this practice of trust in testimony. Thus, building a credulous character has negative social implications as well as negative consequences for the individual.

\section{Conclusion}
We looked at three ways to understand the claim that there is no objective answer to the question whether God exists. Here are some take away points:
\begin{enumerate}
 \item If ``God'' has multiple meanings depending on who is using it, we can still debate whether an entity that meets one of the particular meanings exists.
 \item The truth of the claim ``God exists'' could be relative, but it doesn't seem to be of the same sort as claims that genuinely do appear to be relative.
 \item We looked at one argument (Clifford's) for thinking that we have an obligation to believe only what we have evidence for. If the argument succeeds, then one's belief in God (or non-belief) commits one to having adequate evidence for its truth.  And it isn't enough just to pass it off as \textit{mere opinion}.
\end{enumerate}

As usual, we haven't shown that the question of whether God exists has an objective answer. There may be different ways of understanding the claim that there is no objective answer. Instead, what we have tried to do is better understand what the claim means. When we look closely at some of the possibilities, we see that it isn't so clear that they really do justify the claim that the search for an objective answer is futile.

\end{document}
