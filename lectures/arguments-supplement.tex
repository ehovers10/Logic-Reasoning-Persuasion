\documentclass[10pt]{article}

\usepackage[letterpaper]{geometry}
\usepackage{enumerate,verbatim}
\usepackage{fancyhdr}
\usepackage{linguex}
\usepackage[colorlinks=true,linkcolor=blue]{hyperref}

%Packages needed for trees
\usepackage{amsfonts,amsmath,amssymb}
\usepackage[varg]{txfonts}
\usepackage{qtree}

%% Margin Setting
\geometry{hmargin={.5in,.5in},vmargin={1in,1in}}
\setlength{\parindent}{0.0in}
\setlength{\parskip}{2mm}
\setlength{\tabcolsep}{10pt}
\setlength{\arraycolsep}{10pt}

%% Header
\pagestyle{fancy}
\fancyhead{}
\fancyhead[L]{Phil 101, f14}
\fancyhead[C]{Supplement on arguments}
\fancyhead[R]{Hoversten}

\begin{document}

\textbf{Categorization of arguments}

\Tree [.{\underline{\textbf{Set of propositions}}}
  [.{\textbf{Argument}} 
   [.{\underline{\textit{Deductive}}} 
    [.{\textit{Valid}} 
      [.{Sound} ]
      [.{Unsound} ] ] 
    [.{\textit{Invalid}} ] ] 
   [.{\underline{\textit{Inductive}}} 
    [.{\textit{Strong}} 
      [.{Cogent} ]
      [.{Uncogent} ] ] 
    [.{\textit{Weak}} ] ] ]
  [.{\textbf{Non-argument}} ] ]
  
\paragraph{Everyday prose} In normal communicative exchanges, people are rarely precise about what the premises for their arguments are.  They may take certain claims to be obvious, and thus not explicitly mention them.  But when we're trying to critically assess an argument that someone has given us, it is important to be clear about every piece of the reasoning involved.

So, when we extract an argument from everyday prose and put it into premise/conclusion form, we should be aware that some of the premises may be \textbf{implicit} in the prose as it's written.  That is, we may have to \textit{read between the lines} and add extra premises to our representation of the argument in order to fill out all the pieces of the reasoning involved.

\paragraph{The method of counterexample} Deductive arguments are ones in which the premises purport to give \textbf{conclusive} support to the conclusion.  That means that in a good deductive argument, if the premises are true, then the conclusion \textit{must} be true.  But not all deductive arguments are good ones.  If the premises don't actually provide conclusive support for the conclusion, then we say that the argument is \textbf{invalid}.  When we look more closely at deductive arguments, we will examine different ways of determining whether an argument is valid or invalid.  But one simple way to show that a deductive argument is invalid is to provide a \textbf{counterexample}.

A counterexample is a hypothetical situation in which all of the premises come out true, but the conclusion comes out false.  If we can come up with a coherent scenario like this, then we have shown that the argument didn't do what it set out to accomplish. Since deductive arguments are supposed to show that the conclusion \textit{must} be true, if there is even a far fetched possibility where the premises don't align with the conclusion, then the argument fails; it is invalid.

As an example, consider the following argument:

\begin{enumerate}
  \item All dogs are animals.
  \item Some animals are viscious.
  \item $\therefore$, all dogs are viscious.
\end{enumerate}

This is not a good argument, and we can show that by constructing a counterexample.  Imagine a scenario in which there are just 2 characters:

\begin{itemize}
  \item Gus: is a dog, is an animal, is super friendly
  \item Scar: is a lion, in an animal, is viscious
\end{itemize}

According to this scenario, premise 1 is true because every dog in our scenario is also an animal.  Premise 2 is also true because one of the animals (Scar) is also viscious.  But the conclusion (3) is false because one of the dogs (Gus) is not viscious at all.

Thus, our hypothetical scenario shows that it is possible for the premises to be true without the conclusion also being true, and this is enough to show that the argument is invalid.

\end{document}
