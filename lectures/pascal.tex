\documentclass[letterpaper,10pt]{article}

\usepackage[colorlinks=true,linkcolor=blue]{hyperref}
\usepackage{amssymb}

%Margin settings
\usepackage{geometry}
\geometry{hmargin={.75in,.75in},vmargin={1in,1in}}

%Header settings
\usepackage{fancyhdr}
%\setlength{\headheight}{15pt}
\pagestyle{fancy}
\fancyhead{}
\fancyhead[L]{Phil 101, f14}
\fancyhead[C]{Decision theory}
\fancyhead[R]{Hoversten}

%Paragraph settings
\setlength{\parindent}{0pt}
\setlength{\parskip}{2ex plus .5ex minus .2ex}

\begin{document}

\section{Introduction}
Some people feel that it's impossible to prove the existence or non-existence of God, and that as a result of this, we are free to believe whatever we wish on this matter.  One should be careful here to not fall into the \textit{fallacy of ignorance} where one concludes from the fact that there is no proof of a claim that the claim is false. One must also be careful not to carry this idea too far because there are many things that we don't have proof of, but we don't consider it ok to believe whatever you want.  For instance, I may not have proof that this gun is loaded, but that doesn't mean I can choose to believe that it isn't loaded and point it whereever I want.  Beliefs have consequences for actions, and frequently, it is irresponsible to choose not to believe something even when there is no proof of it.

There is an interesting question where we should stand on the question of God's existence. If reason doesn't determine the truth of the matter either way, then are we free to believe whatever we want?  Certainly, competing beliefs about God have led to a number of sad actions throughout history, so it seems to be a matter that we would like to get settled.

But it often seems that individuals can believe in God without this belief leading to dire consequences regarding their interaction with others.  So, we might conclude that we can't condemn those who believe in God on faith for violating some principle of reason. But often, those who believe in God don't want to just leave it at that.  They also want to convince others that belief is the \textit{right} choice. So, we might wonder if there is some other way to convince those who \textit{don't} have faith that they should believe in God nonetheless.  In presenting his wager, this is what Pascal tries to do.

\begin{quotation}
    God is, or He is not. But to which side shall we incline? Let us see. Since you must choose, let us see which interests you least. You have two things to lose, the true and the good; and two things to stake, your reason and your will, your knowledge and your happiness; and your nature has two things to shun, error and misery. Your reason is no more shocked in choosing one rather than the other, since you must of necessity choose... But your happiness? Let us weigh the gain and the loss in wagering that God is... If you gain, you gain all; if you lose, you lose nothing. Wager, then, without hesitation that He is. (Blaise Pascal, \textit{Pense\`es})
  \end{quotation}

\subsection{Two kinds of reasons}
Pascal's argument attempts to provide a \textbf{pragmatic} reason to believe in God.  We've already discussed the notion of a pragmatic reason, but let's recap the idea again:

\begin{quote}
Imagine that you are being chased through the woods by a bear.  All of a sudden, you break into a clearing and approach the edge of a chasm.  The chasm looks to be about 20ft wide. It occurs to you that if you try to change course, you will most certainly be devoured by the bear.  But you also recall that the best long jumpers in the world rarely jump more than 25ft, and you are no world class athlete. You are in quite a predicament.
\end{quote}

In this scenario, both of the following claims seem to be true.
\begin{enumerate}
 \item You have reason to believe that you cannot make the jump.
 \begin{itemize}
  \item All of your evidence suggests that the distance is too far for you to cross in a single leap.
 \end{itemize}
 \item You have reason to believe that you can make the jump.
 \begin{itemize}
  \item If you don't jump, you're gonna get eaten.  And maybe believing you can make it will give you a slightly better chance of doing so. So, you might as will believe it.
 \end{itemize}
\end{enumerate}

On their face, claims 1 and 2 are in direct contradition. The fact that they both seem true suggests that there is an ambiguity in our use of the word ``reason''.

Let's use the term \textbf{epistemic reason} to refer to the kind of reason mentioned in 1.  Epistemic reasons appeal to our cognitive faculties and purport to support the \textbf{truth} of some claim.  And let's call the kind of reason mentioned in 2 \textbf{pragmatic reason}.  Pragmatic reasons appeal to our connative faculties (our senses of emotion and well-being) and purport to show that we \textbf{benefit} in some way from believing the claim.

We've mostly been concerned with arguments involving epistemic reasons.  But Pascal thinks that these sorts of reasons can't decide the issue of whether to believe in God, and when that is the case, we can look for pragmatic reasons to help help us decide whether to believe.

While pragmatic reasons don't necessarily concern the \textit{truth} of some claim, it doesn't mean that they can't be the subject of rigorous argument.  And a rigorous argument for believing in God is just what Pascal tries to provide.

\subsection{Pascal's argument}
Pascal maintains that we have strong pragmatic reason to believe that God exists. His reasoning is roughly as follows:
\begin{enumerate}
 \item If you believe in God, you have the chance for eternal happiness if it turns out he does exist.
 \item If you don't believe in God, you risk eternal damnation if it turns out he does exist.
 \item The possible reward for belief is huge, and the risk for non-belief is devastatingly high.
 \item So, you should believe in God.
\end{enumerate}

\paragraph{How do we assess this argument?} Notice first that Pascal has given us an \textit{inductive} argument.  The premises don't guarantee the conclusion because the premises themselves are somewhat vague. The main push of the argument seems to be one of \textbf{cost/benefit} analysis. Cost/benefit analysis is a form of reasoning with the following components:
\begin{itemize}
 \item An individual is faced with a choice between $A$ and $B$.
 \item The individual tallies up the pros and cons of both $A$ and $B$.
 \item The individual adjusts the lists above to factor in the $significance$ of the pros and cons for $A$ and $B$.
 \item The individual compares the weighted lists.
 \item Whichever choice comes out ahead is the one the individual decides to carry out.
\end{itemize}
Cost/benefit analysis is a basic form of human reasoning.  It is something that we perform all the time. Let's call the pros and cons of various choices \textbf{outcomes} of that choice.

\paragraph{Measuring the significance of outcomes} An important part of the above reasoning is the adjustment of the outcome lists on the basis of significance.  It isn't very valuable to just list out the pros and cons and add up the number of each, because not all outcomes are of the same value. 

There are two ways that significance can factor in:
\begin{itemize}
 \item Some outcomes are better or worse than others.  
  \begin{itemize}
    \item If I'm looking at two different jobs and both offer 2 perks, but job A offers free coffee and a parking spot while job B offers dental coverage and paid vacation time, I'm obviously going to choose job B.
  \end{itemize}
 \item Some outcomes are more likely than others.
  \begin{itemize}
   \item When we're deciding what to do, we don't always know what will result from certain choices because we don't know exactly how the world is laid out.
   \item So we want to consider all sorts of possible outcomes.
   \item But we can make educated guesses about how likely certain outcomes are, and the more likely ones should count for more in our decision making.
  \end{itemize}
\end{itemize}
What we need is a way of making these ideas about the significance of outcomes more precise, so that we can better determine what the right choice is.  \textbf{Decision theory} is an attempt to do just this.

\subsection{Decision matrices and calculating expected value}

To make it easier to assess Pascal's argument, we will make use of a tool from decision theory known as a \textbf{decision matrix}.  Often, we have to make certain decisions when we are uncertain of how exactly the world is laid out. A decision matrix is used to help determine which of a number of possible actions is the best one, given the ways the world might be.

Along the rows of the matrix, we place the various \textbf{actions} amongst which we need to choose.  And along the columns, we place the various \textbf{states of the world} that we think might possibly obtain.\footnote{The states of affairs that we include should \textbf{exhaust} the relevant possibilities, and they should be mutually \textbf{exclusive}.}  Finally, in each box of the matrix, we put that value (either positive or negative) that we would assign if we chose the relevant action and the relevant state of the world obtained.\footnote{How exactly these values are set is a fairly subjective matter.  We can usually get some valuable information from the matrix as long as the values we use are appropriate relative to eachother.  So, an event that seems really awesome would get a high positive value, and an event that would be a super bummer would get a comparable negative value.}

\begin{center}
 \begin{tabular}{c|c|c|}
 & State 1 ($S_1$) & State 2 ($S_2$) \\ \hline
Action 1 ($A_1$) & $Val(A_1S_1)$ & $Val(A_1S_2)$ \\ \hline
Action 2 ($A_2$) & $Val(A_2S_1)$ & $Val(A_2S_2)$ \\ \hline
\end{tabular}
\end{center}

We can then use this matrix to determine which of the actions before us is the best.  But to do this, we need to assign probabilities to the various states of affairs.  The value we assign to a state of affairs represents how likely we think it is that that state of affairs will obtain.  With all this information established, we can use the following formula to determine the expected utility of choosing each action.  The expected utility is a numerical representation of how good of a choice that action is compared to the other options.

\[EU(A_1)=Val(A_1S_1)\times P(S_1) + Val(A_1S_2)\times P(S_2)\]

After calculating the expected value for each of the actions we have to choose from, the action that gets the highest result is the one that we have most pragmatic reason to choose.

\subsection{Do we have pragmatic reason to believe that God exists?}
Let's apply these tools to the choice of whether or not to believe in God.  Below is how Pascal would set up the decision matrix.

\begin{center}
 \begin{tabular}{c|c|c|}
 & God exists & God doesn't exist \\ \hline
Believe in God & $+\infty$ & \textsc{finite} \\ \hline
Don't believe in God & $-\infty$ & \textsc{finite} \\ \hline
\end{tabular}
\end{center}

The value in the top left box represents your admission into Heaven when you believe and God exists.  The value in the lower left box represents the eternal damnation that a non-believer receives when God exists.  The finite values in the right column represent the fact that, if God doesn't exist, the positive or negative value you receive is limited to what happens while you're alive. As we'll see, it doesn't really matter what we set these values at, so long as they are finite.

To start with, let's assume that we have no epistemic reason one way or the other regarding belief in God.  That means that the probability of each state of affairs is split 50/50.  Thus, we calculate the expected value of each option:

\[\begin{array}{rcl}
EU(Belief) & = & Val(Believe,God)\times P(God) + Val(Believe,no \; God)\times P(no\;God) \\
 & = & +\infty\times 0.5 + \textsc{finite}\times 0.5 \\
 & = & +\infty \\   
  \end{array}\]

\[\begin{array}{rcl}
EU(non-Belief) & = & Val(no\;Believe,God)\times P(God) + Val(no\;Believe,no \; God)\times P(no\;God) \\
 & = & -\infty\times 0.5 + \textsc{finite}\times 0.5 \\
 & = & -\infty \\   
  \end{array}\]

Thus, since positive infinity is vastly better than negative infinity, this setup suggests we have good pragmatic reason to believe in God.

\subsection{Objections}
While the argument is pretty straightforward, there are some complications.  Let's look at some objections to the argument that are usually raised.  The first three are common, but Pascal can offer an answer to them.  The last two are a bit more serious.

\begin{enumerate}
 \item It's not enough for someone just to believe in God for them to get into Heaven.  They must also do good deeds, go to church, and stuff like that.
 \begin{itemize}
  \item This may be true, but it doesn't really impact Pascal's argument.  After all, it's still necessary for someone to get into Heaven that they believe in God. So, you still have reason to believe, even if you must do other stuff as well.
 \end{itemize}
 \item One can't just choose to believe something.  Whether you genuinely believe something depends on stuff like the evidence you have. Just because you want to believe something doesn't mean you genuinely do believe it.
 \begin{itemize}
  \item This is also true, but if you recognize that you have pragmatic reason to believe, you can do certain things to help you develop the belief in God.  For instance, you can go to church and surround yourself with other believers.  If you work at it, you will likely engender a belief in God.
 \end{itemize}
 \item Pascal only considers the choices of believing in God or believing God doesn't exist. But don't we also have the choice of withholding belief entirely?
 \begin{itemize}
  \item It is true that for many claims it is a genuine third option to withhold belief. For instance, I neither believe nor disbelieve that there are an even number of particles in the universe.
  \item But in the case of belief in God, withholding belief is equivalent to disbelief.  At least according to the model Pascal is working with, agnostics suffer the same fate as atheists.
 \end{itemize}

 \item Pascal has already admitted that reason can't tell us anything about God.  Therefore, we can't know what his will is.  Maybe God punishes those who believe based on pragmatic arguments.  That is, he only accepts those who come to believe in him for the right reasons.  If so, then the values we placed in the decision matrix might not be justified.

 \item Pascal only considers the states of affairs in which the Christian God exists, but there are lot's of different religions out there.  
 \begin{itemize}
  \item This fact is not in and of itself problematic for Pascal's argument because the existence of different religions doesn't change the values we put in the boxes of our decision matrix.  But it does suggest that our matrix might be incomplete.
  \item Imagine the possibility that a different god exists.  Call this god ``Anti-God''.  Anti-God punishes anyone who believes in God by sending them to Hell and accepts anyone who doesn't believe in God into Heaven.  If we consider this possibility, our decision matix should look like:

\begin{center}
 \begin{tabular}{c|c|c|c|}
 & God exists & no God exists & Anti-God exists \\ \hline
Believe in God & $+\infty$ & \textsc{finite} & $-\infty$ \\ \hline
Don't believe in God & $-\infty$ & \textsc{finite} & $+\infty$ \\ \hline
\end{tabular}
\end{center}

  And our new expected values look like:\footnote{In the following calculation, I set the probabilities of each state of affairs to be equal, but really, it doesn't matter what we put them at so long as each is more than zero. The infinities will swamp everything out.}

\[\begin{array}{rcl}
EU(Belief) & = & Val(Believe,God)\times P(God) + Val(Believe,no \; God)\times P(no\;God) + \\
 & & Val(Believe,anti-God)\times P(anti-God)\\
 & = & +\infty\times 0.33 + \textsc{finite}\times 0.33 + -\infty\times 0.33\\
 & = & \textsc{finite} \\   
  \end{array}\]

\[\begin{array}{rcl}
EU(non-Belief) & = & Val(no\;Believe,God)\times P(God) + Val(no\;Believe,no \; God)\times P(no\;God) + \\
 & & Val(no\;Believe,anti-God)\times P(anti-God) \\
 & = & -\infty\times 0.33 + \textsc{finite}\times 0.33 + \infty\times 0.33 \\
 & = & \textsc{finite} \\   
  \end{array}\]

  So now it appears that both options are equally beneficial.  The upshot is that so long as the existence of Anti-God is a genuine possibility, we don't really have a pragmatic reason for believing in God.
 \end{itemize}
\end{enumerate}

\section{Conclusion}
We've examined one way of trying to justify belief in God from a pragmatic, as opposed to epistemic, perspective. In the end, we've found Pascal's wager to have a couple holes in it.
\begin{itemize}
 \item This doesn't mean that there are no pragmatic reasons to believe in God, only that this way of arguing for it is unconvincing.
 \item Nor have we shown that believing in God based on faith is illegitimate. The question of the justification of faith is a very interesting one that we won't be able to examine in this class. If this is something that interests you, you might consider taking another course on the philosophy of religion and religious experience.
\end{itemize}

\end{document}
